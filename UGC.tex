\chapter{Acknowledgements}
\textit{First, we would like to thank the Tow Center for Digital Journalism at Columbia University for supporting this project—Emily Bell, Taylor Owen, and Lauren Mack in particular. With their help, a crazy idea over a Skype call has become something that will actually provide the news industry with concrete numbers about its use of user-generated content. We hope the research will act as a foundation for conversations and will provide support and suggestions for best practices. We would also like to thank the international cooperation team at NHK in Tokyo for recording three weeks of its broadcasts for us. John, Sam's stepfather, was another who helped us by recording CNN when we had lost all hope of capturing it from our respective locations in Istanbul and Málaga, Spain. We would also like to thank many different people for conversations we've had over the last eight months at conferences, in coffee shops, and down the pub. It feels like we've been thinking about and talking about UGC constantly for the past year. Every one of those talks has shaped an aspect of this report. And, finally, thanks to all those who took time out of their schedules to talk to us about UGC and how they use it in their news bulletins. The interviews were newsworthy and thoughtful throughout.}

\chapter{Executive Summary}
\title{Aim of Research}
\begin{The aim of this research is to provide the first comprehensive report about the use of user-generated content (UGC) among broadcast news channels. Its objectives are to understand how much UGC is used on air and online by these channels, why editors and journalists choose to use it, and under what conditions it is employed. The study intends to provide a holistic understanding of the use of UGC by international broadcast news channels.
Methodology
This research was carried out in two phases. The first involved an in-depth, quantitative content analysis examining when and how eight international news broadcasters use UGC. For this part of the research we analyzed a total of 1,164 hours of TV output and 2,254 Web pages, coding them according to parameters intended to answer the research questions. The second phase of the research was entirely qualitative. It was designed to build upon the first phase by providing a detailed overview of the professional practices that underpin the collection, verification, and distribution of UGC. To achieve this we conducted 64 interviews with news managers, editors, and journalists from 38 news organizations based in 24 countries around the world. This report brings together both phases of the research to provide a detailed overview of the key findings.

\title{Research Questions}
\begin{This research was designed to answer two key research questions: 
\begin{enumerate}
1. When and how is UGC used by broadcast news organizations,
on air as well as online?
2. Does the integration of UGC into output cause any particular
issues for news organizations? What are those issues and how do
they handle them?
\title{Principle Findings}
The key findings from our content study were:
\begin{itemize}
\item UGC is used by news organizations daily and can produce stories
that otherwise would not, or could not, be told. However, it
is often used only when other imagery is not available.
\item News organizations are poor at acknowledging when they are
using UGC and worse at crediting the individuals responsible for
capturing it. Our data showed that:
\item 72 percent of UGC was not labeled or described as UGC.
\item Just 16 percent of UGC on TV had an onscreen credit.
\item There are more similarities than differences across television and
Web output, but troubling practices exist across both platforms.
\item The best use of UGC was online, mostly because the Web provides
opportunities for integrating UGC into news output like
live blogs and topic pages.
\end{itemize}

The key findings to emerge from our interviews were:
\begin{itemize}
\item News managers are often unaware of the complexities
involved in the everyday work of discovering, verifying, and
clearing rights for UGC. Consequently, staff in many newsrooms
do not receive the training and support required to develop
these skills.
\item With newsrooms under ever-increasing pressure, it is important
that there are systematic procedures in place to provide
clear guidance to output editors about which checks have been
completed, and the level of confirmation regarding specific facts
about footage.
\item There is a significant dependence on agencies to discover and
verify UGC. Many newsrooms, particularly national news organizations,
receive their UGC solely from agencies; often unaware
of the content's origin, they don't realize that they are even using
UGC, and think of it all simply as ``agency footage.''
\title{Conclusions}
\item The rise of UGC means many journalists' roles will change.
Rather than being the sole bearers of truth, journalists will be
required to allow more space for people to tell their own stories
directly. News organizations must therefore face up to the challenge
of deciding how best to manage this change.
\item Crediting practices need to improve; it will not be long before
an uploader takes a news organization to court for using content
without permission or for failing to attribute due credit. The
result of any such case would have wide-reaching implications
for the news industry.
\item As high-value UGC is increasingly licensed in the immediate
aftermath of breaking news events, newsrooms should get used
to paying for this content.
\item When putting out calls for action, newsgatherers need to use
language that leaves uploaders without doubt that they must not,
under any circumstances, put themselves at personal risk for the
sake of capturing newsworthy UGC. Where required, training
should be provided in this area.
\item News managers very quickly need to understand the full implications
of integrating UGC into their output, with regard to its
impact on their staff, their audiences, and the people who are
creating the content in the first place.
\item Given the strong reliance on agencies to discover and verify
UGC, news managers need to gain a stronger understanding
of the practices employed by different agencies. This will
enable them to ask appropriate questions about the provenance
of a piece of UGC and the verification checks that have
been undertaken.
\item The issue of vicarious trauma among staff who work with UGC
is beginning to receive recognition as a serious issue, and news
organizations must strive to provide support and institute working
practices that minimize risk.
\end{itemize}

\chapter{Introduction}
It is just over five years since US Airways flight 1549 landed in the Hudson
River in New York City. In the immediate aftermath, an incredible photograph
emerged of passengers crowding onto the wing, awaiting rescue. It
originated from the Twitter account of Janis Krums, who had tweeted it out
to his 60 or so followers. At that point, only about 10 journalists worldwide
knew how to find that picture, how to verify it, whether they needed to seek
permission to use it, and whether they had the right to put the picture on
air or online. In the five years since 2009, newsgathering around breaking
news events has been revolutionized by the pictures and videos captured
by eyewitnesses uploaded to social networks. The news industry has been
running to catch up with people's behavior around news events ever since.
The speed at which newsgathering has changed is astonishing. One journalist,
who works on a UGC desk, admitted hearing people once say, ``Why
would we want to use this? Look at the quality of mobile footage; who would
be interested in it? Now when a major story happens, everyone is beating at
the UGC door. We're the first port of call.''
While citizen journalists, or nonprofessionals with an interest in documenting news events, have taken some of these pictures and videos, indeed many have simply been shot by ``accidental journalists''—people with a camera or smartphone on hand who happened to be in the right (or wrong) place at the right (or wrong) time.

Rarely do these people recognize the value of their footage. Instead of contacting news organizations directly, they want to share what they have seen with friends and family via the social Web. As Anthony De Rosa (ex-social media editor for Reuters and now editor-in-chief for Circa) writes, ``The first thought of the [uploader] is usually not: ‘I need to share this with a major TV news network,' because they don't care about traditional television news networks or more likely they've never heard of them. They have, however, heard of the Internet and that's where they decide to share it with the world.''^{\href{#endnotes}{1}} Amateur content-capturers have their own audiences to think
about now.
Still, the news media is able to find UGC, and they do so in droves. Although
this revolution has been acknowledged, no hard numbers exist about the
amount of UGC being used by broadcasters. Furthermore, while journalists
at conferences talk about these processes publicly, there haven't been any—
even off-the-record—conversations we've been privy to about the reality of
handling UGC in a breaking news context.
The other area we were compelled to explore involved journalists' perception
of UGC. Previous academic studies have concluded that journalists
simply consider UGC as another source. These studies demonstrate that,
for the most part, journalists do not view the integration of UGC as a participatory,
collaborative activity. Instead, they set the agenda and use content
supplied by their audiences when they feel it is relevant.
Our research was therefore designed around two core questions:
\begin{enumerate}
1) How is UGC used by broadcast news organizations, on air as
well as online?
2) Does the integration of UGC into output cause any particular
issues for news organizations? What are those issues and how do
they handle them?
\end{enumerate}
To answer these questions, we sampled eight 24-hour news channels. We
recorded 1,164 hours of television output, and captured 2,554 Web pages
over a three-week period at the end of 2013. We then systematically analyzed
the amount of UGC integrated into output on-air and online, and examined
when and how UGC was used. We combined this quantitative analysis with
64 semi-structured interviews with journalists, editors, and managers at 38
news organizations (rolling news channels as well as national news outlets)
located in 24 different countries.
Overall, the majority of the 40 newsrooms included in this study, located
all across the world, use UGC in their output. For bigger newsrooms, especially rolling news channels, UGC features almost daily. Crucially, UGC is used when other images are not available, either from a newsroom's own
staff or the news agencies with whom they contract. However, there is a
very small, but increasing number of newsrooms that see the benefit of
investing in UGC to tell different or better stories—and this is starting to
have an impact.
There is a very significant reliance on news agencies to discover, verify,
and clear the rights for UGC—especially for foreign stories. The number
of newsrooms that have dedicated staff for these processes is still
relatively small.
Most interesting, perhaps, is how disconnected most news managers are to
the specific processes associated with the integration of UGC. Those who
work with UGC on a daily basis discuss its integration in a significantly different
way than do their managers.
The journalists we spoke to about the integration of UGC raised six main
areas of concern:
\begin{enumerate}
1) WORKFLOW: Should newsrooms have staff dedicated to
the processing of UGC, or should the task be shared across
the newsroom?
2) VERIFICATION: Verifying UGC is considered a most pressing
challenge, particularly in the pressured context of breaking news.
3) RIGHTS CLEARANCE: The legal issues associated with copyright
law concern everyone. Broadcasters that extract UGC from
social networks in order to use it on air worry about breaking
terms and conditions, whereas online journalists worry about
embedding content without seeking permission from its creator
(which is actually permissible according to the terms and conditions
of the social networks).
4) CREDITING: Debates exist about whether on-air crediting is
necessary, with the added complication that some news
agencies that supply UGC do not provide any information about
the uploader.
5) LABELING: Labeling UGC is also a concern. While accidental
journalists, or eyewitnesses with camera phones, create some
of the UGC used by broadcasters, people with a specific agenda
film a great deal more. That could be an activist group in Syria
or an aid worker in the Central African Republic. Newsrooms
know that for reasons of transparency it is important to
label UGC, but they are not sure how to do this appropriately
and consistently.
6) RESPONSIBILITIES: In the specific context of UGC, the ethical
responsibilities newsrooms must uphold for the uploaders,
the audience, and their own staff are numerous.
\end{enumerate}
These six areas of concern form the backbone of the way this report is organized.
The strength of each of these as standalone topics encouraged us to
create six digestible mini-reports.

One major tension runs throughout the research, however. That is the
debate about the role of the journalist as gatekeeper. A handful of interviewees
talked about UGC as a crucial way of strengthening the relationship
of newsrooms with their audiences, but were also very honest about
how technology was threatening the established role of journalists. As one
interviewee admitted, ``The crowd will many times think of something better
than you do. And I think that's something we refuse to believe because it
shakes the foundations of everything we [do] as gatekeepers.''
The view that journalists should remain the bearers of truth permeated our
interviews: ``We should be gathering as much as we can ourselves. Our job
is to be the eyewitnesses for people who can't be there, to assimilate and
disseminate facts, and to separate the truth from the untruth.''
As another interviewee argued, ``Why are we journalists? Why did we
become journalists? To make other people do the storytelling? We have
to be very judicious in the way we use this stuff and not let it take over
the story.''
It is important to note that a handful of the UGC that featured during the
three-week period was actually the inspiration for certain stories, which
otherwise would not have been told. The ability of UGC to highlight illegal
practices, or to illustrate stories that could not appear without pictures, was
rarely discussed in our interviews.
For the most part, the audience is thought about as a potential source for
breaking news pictures once a newsroom has been alerted to a story and
has decided to run with it. The audience is not often considered a partner
in producing compelling content. There were a few noticeable exceptions
among our interviewees, who acknowledged that technology allows people
to tell their own stories. These journalists argued that this should not be
considered a threat; that journalists are professional storytellers, and their
role will always be to provide the necessary layer of verification, context,
and a narrative framework. This tension will continue to impact discussions
about UGC, and it is unsurprising that it runs throughout this research.

\chapter{Definition of User-Generated Content}
The phrase ``user-generated content'' is a catchall that can mean different
things to different people, even those working in the same newsroom. For
the purposes of this study, we define UGC as photographs and videos captured
by people who are not professional journalists and who are unrelated
to news organizations. It does not include comments (either posted underneath
a news article or posted to social networks) integrated into coverage.
What We Didn't Find
Before we launch into our main research discoveries, we wanted to highlight
what we found in relation to some of the assumptions commonly shared
about the integration of UGC: that it is only used because it is a cheap way
to source pictures and that news is being dumbed down by viral videos of
talented pets and amusing babies.
Neither of these assumptions is true. Managers shared with us the cost of
resourcing the integration of UGC into their output, in terms of discovery,
verification, and clearing rights. Not one newsroom considered UGC a
cheap alternative.
In terms of viral video, over the three weeks of our sample period, there
was only one video that could be described as ``viral'' and it only appeared
online. Not only were viral videos almost nonexistent, but there was also a
very low percentage of weather content (commonly thought of as a news
magnet) used by the broadcasters we studied. So the assumption that UGC
is all about weather and pets appears not to be true. Broadcasters might
receive the most content around these subjects, but they're not using it
very much.

\section{How to Read the Report}
The long-form platform Medium advises you about how many minutes each
article on its site will take to read. If you're going to sit down and read all six
mini-reports here, it will certainly require quite a lot of minutes. However,
we hope that you will dip in and out of different sections, and share those
that you think are most relevant.

\chapter{Brief History of UGC}
As the BBC's deputy director of news and current affairs, Fran Unsworth,
reminds us, ``UGC is nothing new,'' it's just ``much more prevalent than it
ever was because everybody has a camera.'' Frank Zapruder's film of the
assassination of President Kennedy and George Holliday's shaky footage of
the beating of Rodney King remind us that eyewitness pictures were newsworthy
long before the famous picture of the plane landing on the Hudson
river—the one often used as the definitive example of UGC.
The history of user-generated content within the mainstream broadcasting
context usually begins with the London bombings in July of 2005. This was
the first time the BBC led a bulletin with imagery not filmed by a BBC camera,
using pictures captured by people escaping the scene via underground
tunnels. It was also the news event that convinced the BBC to establish
its UGC Hub, then just a pilot project, as a permanent fixture within
the newsroom.
However, according to Patricia Whitehorne, who was part of that very small
UGC project, the tsunami on December 26, 2004 was the first occasion on
which UGC was sought in a systematic way. ``That was the first time that the
News Channel [then News 24] was beating on our door, saying, ‘We need
photos, we need eyewitnesses, we need emails.' They sent a correspondent
to do a package about the UGC that was coming in from the tsunami, and I
think that was the first time they really saw the value, or the potential of it.''
However, Whitehorne admitted that it took a while for attitudes to change.
``I remember right at the beginning, having to go to different editorial meet
ings, just trying to convince people of the value of UGC, explaining what it
is. There always used to be a joke, that whenever we said UGC, they'd say,
‘Isn't that a chain of cinema?' It took a lot of explaining.''
In terms of understanding the trajectory of newsroom attitudes toward
UGC, numerous interviewees cited the Iranian protests in June of 2009 as
a watershed moment. John Ludlum from Reuters said, ``That was one of the
first times that we started to see [UGC] and took tentative steps towards
using it.'' Mark Little, then a foreign correspondent at the Irish public broadcaster
RTÉ, talked passionately of his personal frustration at not being sent
to Tehran to cover the election in 2009. He quickly realized, however, that
he was able to access vast amounts of information via Twitter and YouTube,
although he struggled to know what to trust. As a result he left his job at
RTÉ and founded the social news agency Storyful.
According to most of our interviewees, 2011 was the year that UGC went
mainstream across newsrooms. As Chris Hamilton, manager of the BBC's
UGC Hub, explained:
[The year] 2011 was a very big watershed. Obviously you had the
Arab Spring getting under way, but that was also the same year as
the Japan tsunami, the riots in England, and the massacre in Norway.
All of those were big, really mass participation events… and news
organizations were able to take advantage of that to tell those stories
better than they could have told them before.
Our research revealed the Syrian conflict as the main impetus for UGC use
during the three-week period we studied. In fact, 40 percent of all the UGC
we analyzed during our sample period was connected to Syria. And for
some organizations, it was the only story for which they integrated UGC.
Peter Barabas, editor-in-chief for news at euronews, admitted that his organization
has only ``started using [UGC] dramatically over the past two years.
The war in Syria made it very clearly a necessity because there is no way for
us to cover Syria other than UGC.''

And certainly, learning how to use UGC from Syria has had an inevitable
impact on the way UGC is used for other stories. As Geertje Bal from the
foreign desk at VRT in Belgium explained, ``People are using UGC more
for other stories because the Syria conflict opened the way. In the past you
would say, ‘No, no, no, we don't do that with amateur material, but it has
become more accepted because we had to do it for Syria.''
The impact of mobile-phone penetration, with higher quality built-in cameras,
combined with improvements in connectivity, cheaper data, and the
fact that social networks are still growing in popularity globally, means
usable eyewitness footage will simply become more of a regular occurrence.
Derl McCrudden, head of newsgathering for AP Television News, explained
this cycle:
I think the use of UGC has inevitably gone up because we're in a more
connected world. More of us have phones… and therefore more people
become accidental eyewitnesses to events. And the more that
happens, the more demand there is for the content because you capture
what people want. So the more we look for it, and then filter it,
and verify it and make sure that it's good to go, the more we therefore
put out incrementally.
A significant development in the past couple of years is journalists filming
content themselves on their smartphones and uploading it to social networks.
One news organization admitted deciding it needed a specific term
for this type of content, so now call it JGC—journalist-generated content.
There is also a growing phenomenon of aid workers and field staff using their
phones to create content to share widely on social networks. By capturing
pictures of refugees crossing the Jordanian border, or people sheltering at
Bangui M'Poko Airport in the Central African Republic, they know these
pictures will be seen by their own supporters and may drive fundraising.
There's also an awareness that news organizations are looking for this type
of footage, particularly from places where they are struggling to send their

own reporters. This content type is included within our broad definition of
UGC, as the pictures are available on personal social media accounts. However,
in the same way that our interviewees discussed the need to be transparent
with agendas associated with activist groups' uploaded video from
Syria, the humanitarians capturing these pictures are not accidental journalists.
They have their own motivations, and this should also be explained
to the audience.
Definition
The phrase user-generated content has always been an unpopular one.^{\href{#endnotes}{2}}
However, no one has managed to create an alternative that adequately
describes the phenomenon. In research conducted by one of this report's
co-authors about UGC in 2008, a typology of five different types of UGC
was developed,^{\href{#endnotes}{3}} which differentiated breaking news footage from community
journalism initiatives and user-generated opinion and comment.
Again, for the purposes of this study, we define UGC as photographs and
videos captured by people who are not professional journalists and who
are unrelated to news organizations. It does not include comments (either
posted underneath a news article or posted to social networks) integrated
into coverage.
This means statements posted on social networks by newsmakers (e.g.,
celebrities, politicians, sports people, or institutions like the United Nations)
that are using social networks to bypass traditional public relations channels
are not classified as UGC. So, for example, a golfer posting a picture
of a new set of clubs he's received from his sponsor does not qualify. On
the other hand, a picture tweeted by a soccer player of himself watching
the 2014 FIFA World Cup draw with his teammates is included, as it is not
classified as P.R. Additionally, pictures or footage shot by an individual aid
worker would be included (although, in fact, no examples of this appeared
in our sampled timeframe).

\chapter{Methodology}
This research examines when, how much, and within what parameters
user-generated content is used by global broadcast news organizations,
both on air and online. We began the project by interviewing 10 news managers
about the integration of UGC within their newsrooms, to provide
a framework for our research questions. We then collected and analyzed
three weeks of output on air and online from eight 24-hour news channels
in order to provide us with concrete numbers about the amount, timing,
labeling, and crediting of user-generated content. Both the television and
online content were coded using predefined characteristics that correlated
with objectives to understand when and how UGC is integrated, how it is
described to the audience, and whether the uploader is acknowledged and
credited. Each individual piece of UGC was analyzed for these characteristics.
We presented our original quantitative data set and analysis in a standalone
Part I report—but the main findings are included in this final report
as well.
The second phase of our research involved extended interviews with 64 news
managers, editors, and journalists from 40 news organizations, located in
24 countries. These interviews allowed us to understand in greater detail
the editorial parameters and the most significant challenges that broadcast
newsrooms are facing with the integration of user-generated content.

\section{Sample}
Our sample included eight 24-hour news channels. This was a deliberate
decision, as we knew anecdotally that national half-hour news bulletin programs
do not use UGC as commonly as do large, international rolling news
channels; we were already concerned that we would be watching hours
of footage and find relatively small amounts. The sample was designed to
include channels from around the world with an international audience (the
target audience of the channel was cross-border, which therefore excluded
major 24-hour channels like Sky News in the United Kingdom, for instance).
Our intention was to analyze seven full days of output—168 hours from
each of the eight channels, totaling 1,344 hours.
The inbuilt repetition of rolling television news meant we didn't want to
analyze 24 hours of output from the same day, so we sampled eight hours
from each day for 21 days. We also rotated the start time, so on the first day
we recorded from 8 a.m. to 4 p.m. GMT, on the second day from 4 p.m. to
midnight GMT, and on the third from midnight to 8 a.m. GMT. This pattern
was repeated for the 21 days. Recording began on Monday, November
25, and ended on Sunday, December 15, 2013.


Of the eight channels, surprisingly, none of them collect and archive their
output (apart from the BBC for internal purposes). We therefore had to
record the channels as they were broadcast. We achieved this through a
variety of methods, but it did result in some outages caused by power cuts
or live streams dropping out in the middle of the night. As a result, our final
sample was 1,164 hours and 10 minutes (87 percent of our original target).
We analyzed all 21 days of the recordings for all channels apart from Al
Jazeera Arabic. For Al Jazeera Arabic we coded five days, chosen at random,
from the 21-day sample. The reason for this was the difficulty of coding
Arabic output without knowledge of the language. It was impossible
to check whether the presenters or reporters were describing UGC in a
particular way, or whether the captions on screen were relevant. The use of
UGC was also significantly greater than any other channel we coded, and,
in fact, even during five days of output from Al Jazeera Arabic there were
more instances of UGC than from any other channel except Al Jazeera
English over the 21 days.
All eight websites were captured at 6 p.m. (local time for the location of
their headquarters), for all 21 days. Only five days were analyzed,^{\href{#endnotes}{4}} the same
five random days chosen for the Al Jazeera Arabic analysis. This was due to
the sheer amount of content on each site. In total, we analyzed the content
of 2,254 Web pages. The average number of links on each homepage every
day was 56, although there was great variance. NHK World, for example,
only contained links to an average of 13 news stories. CNN International,
on the other hand, linked to an average of 119 stories from its homepage.
Some of these stories contained up to 11 three- to five-minute videos on a
single page, all of which had to be combed for UGC.

\section{Data Collection and Analysis}
The majority of content was not explicitly labeled as UGC, so we had to
investigate many individual cases to confirm that it was user-generated content. This was achieved by cross-referencing content with items available on YouTube, Twitter, Facebook, or Instagram, as well as cross-referencing with photos and videos on the Reuters, AP, or Storyful portals. Because of these challenges, we had to ensure that the three of us were consistent in the way we analyzed the output. Therefore, the content analysis only began after we had reached a 95 percent agreement during pilot coding sessions. Even as the analysis was happening, there was continuous dialogue between all
three researchers about examples that raised questions or issues.
When the content's origin was still unclear, we held a group discussion
about the photo or video. Some of the content from more remote locations
often looked at first glance like UGC, but under closer inspection was often
shown to be footage captured by a local news channel with less sophisticated
video equipment and then distributed by one of the main television
news agencies to its clients. One of the best clues that a piece of content
was filmed by a professional was the raw skill of the camera operator. Often,
professional skills could be identified—such as the way the camera panned
slowly across the action rather than the quick, jerky, or uneven movements
associated with camera-phone video taken by amateurs. Ultimately, the
researchers worked as hard as possible to ensure consistent and accurate
coding, but we acknowledge there is undoubtedly a small margin of error
resulting from the difficulty of coding unlabeled UGC.

\section{Collection and Analysis of the Qualitative Data}
In total, we interviewed 64 people from 38 news organizations based in 24
countries. These people included heads and deputy heads of news, heads of
newsgathering or new media, foreign editors, social media editors, bureau
chiefs, as well as journalists and producers. (A list of all interviewees is
included at the end of the report.)
We interviewed representatives of all organizations coded in Part I of the
research with the exception of editors from TeleSur and Al Jazeera Arabic,
who did not respond to our requests for interview.
We traveled to two news industry conferences—News Xchange in Marrakech,
Morocco in November of 2013 and the EastWest Center Media
Conference in Yangon, Myanmar in March of 2014. These two conferences
enabled us to reach a wide variety of global editors and journalists—reach
that would have otherwise been impossible—including the Waziristan
bureau chief of a Pakistani news organization; editors from Singapore,
Indonesia and the Philippines; and heads of news from Australia, Canada,
and Japan. We also spent one week interviewing journalists in London,
enabling us to gain key insights into the workflows of large organizations
such as the BBC and CNN, as well as the two main television news agencies,
AP and Reuters.
The remaining interviews were conducted over Skype or telephone
with our own industry contacts, or people recommended to us in
other discussions.
The interviews were conducted and analyzed as a sequel to the first, quantitative
part. The interview questions inquired into the decision-making
processes behind the use, labeling, and crediting of UGC content, as
well as the discovery and verification of UGC, workflows, training, and
vicarious trauma.

\chapter{How, When, and Why UGC is Integrated into News Output}
The research showed that UGC is used across the 24-hour news industry
on a daily basis. As Richard Porter, controller of BBC World (English),
explained, ``It has become a central element of the newsgathering process
now. No question about that.''
Throughout our sampling period, all channels used content from activist
groups to report the Syrian conflict.^{\href{#endnotes}{5}} Indeed, for some news organizations
Syria was the only story that included any type of UGC, and our interviewees
emphasized the news organizations' total reliance on content from Syria
because of the difficulties in using their own correspondents. However, our
findings also highlighted a dependence on UGC at times when other pictures
weren't available, such as in the very early stages of breaking news
stories. For example, a helicopter crash in Glasgow happened very late at
night (GMT) on November 29, 2013, and UGC featured heavily as the story
broke. As professional pictures from their own camera crews or news agencies
appeared on Saturday morning, broadcasters chose to update their
packages and reports with these, replacing the UGC.
However, there were also a number of instances when stories ran only
because there was UGC to provide imagery—for example a story about
police brutality in Egypt, which included a secret recording in a police cell.

Many interviewees framed UGC as something to use when nothing else is
available, whereas others saw it as an invaluable resource for keeping stories
alive, discovering different angles, and guaranteeing a diversity of voices
and perspectives.
\section{How Much is UGC Used on TV and Online?}
While 21 days of television content were analyzed from seven channels,
only five days of output were evaluated from Al Jazeera Arabic, so the data
has to be compared separately. In addition, only five days of Web content
were analyzed, so again this must be considered separately.
As TABLE 2 demonstrates, an average6 of 11 pieces of UGC were used every
day on television by news organizations. The average length of a piece of
UGC on screen was 11 seconds. It is evident from this table how much
Al Jazeera Arabic employs UGC, compared to the other channels. Its daily
average number of pieces of UGC was 51 (compared to 11 for the other
seven channels), and the average length of a piece of UGC was 16 seconds
(compared to 11 seconds for the other seven channels).


TABLE 3 on the next page demonstrates a significant range in terms of
how different channels use UGC. Focusing on the television output, Al
Jazeera Arabic used the most UGC in the period sampled with a daily average
of 51.4 pieces. (If the daily average was multiplied over 21 days, there
would have been 1,079 pieces of UGC over the 21-day period on Al Jazeera
Arabic alone).

In terms of Web output, CNN included the most UGC online. It is important
to stress, however, that this is partly because different websites had different
numbers of links on each homepage. CNN had the most links out to stories
from its homepage, with a daily average of 119, compared with 65 links to
stories on BBC World and 13 links to stories on the NHK homepage.

The news websites had specific design features that encouraged depth and
breadth in reporting. As one digital editor explained:
There's the output between TV and there's the output between
digital. And I split those in two because the restrictions, the permissions,
and all the rest [associated with both] can be very, very
different. And that's the great thing about digital. It's much more
collaborative, you know. Embedding, etc., Twitter, photo expansion,
that sort of thing, actually allows you to use a lot more UGC in a
much more natural way.
On television, over the 21 days, 73 stories were told using some element
of UGC. On the Web, over just five days, 115 individual subjects included
at least one piece of UGC. Intuitively, this variance can be explained by
the structural differences between television and the Web explained above.
Online news needs to be refreshed and updated constantly and has near
unlimited space in terms of its different Web pages, whereas television fills
a limited, and immovable amount of time.
We certainly saw the way that live blogs^{\href{#endnotes}{7}} and topic pages provided opportunities
to use UGC in a very different way than television. Live blogs are
featured on Al Jazeera English, BBC, CNN International, euronews (which
used an embedded Storify as a form of live blogging), and France 24. FIGURE
1 presents an example from BBC World from December 6, 2013, called
``UK Tidal Surge: As It Happened.'' It shows the impact of a serious storm on
the United Kingdom. Within this one story there are 22 separate pieces of
UGC (one television package that includes a UGC video,8 six photos from
Twitter, and 15 photos emailed directly to the BBC). Television could not
have included this level of depth in its coverage of the storm.

FIGURE 1: ``UK Tidal Surge: As It Happened,'' BBC World, December 6, 2013
The other format we investigated was topic pages. These pages had curated
content around a similar topic displayed in one place. For topics like Syria,
formats like this on the Web provide far more flexibility in terms of storytelling,
and allow for more context and explanation. BBC World and CNN
International had topic pages for Syria. FIGURE 2 shows an example of one
built by CNN International, specifically exploring the refugee experience
of Syrians.

FIGURE 2: CNN International Topic Page—``Crisis in Syria: The Refugees''
TABLE 4, however, shows that the main story types were similar.


On the Web, the most common category of news employing UGC was
``other.'' During our analysis, the Web ran a number of feature stories, like
``Pictures of the Audience Dressed as Doctor Who Characters'' or ``Pictures
People had Taken During Vacations to North Korea,''^{\href{#endnotes}{9}} that were inundated
with UGC.
Interestingly, one interviewee argued that although the Web facilitates
more opportunities to make creative use of UGC, they're not always taken:
I think that there's still a big gap between digital and TV in the sense
that there's a whole different approach to the ways user-generated
content is thrown up onto TV and used as elements in packages.
And I think in some cases it's good and some cases it's not good. But
interestingly and ironically I think the digital side in a lot of news
companies isn't as quick to adopt. They sort of wait until after it's
used by TV and then get whatever is packaged or put together by
television and then they put it out.
\section{For Which Stories is UGC Used?}
Perhaps one of the most surprising statistics was the relative absence of
content around weather. People are often quick to dismiss UGC as simply
something used to illustrate serious climate conditions. Apart from
severe storms in the United Kingdom and the United States during the coding
period, both of which prompted some UGC usage, especially online,
weather-related UGC did not feature heavily. The weather-related disaster
of Typhoon Haiyan in the Philippines was classified as an individual story,
not a weather story.
Another surprising finding from this analysis was the absence of viral videos.
A viral video, one that reaches over a million views on YouTube in a
short period of time, typically involves a talented toddler, a cute animal, or
a jaw-dropping sports stunt. In the period sampled, none of these types of

videos appeared online or on air. With the success of sites such as BuzzFeed
and Upworthy, the power of viral videos to drive traffic has been well documented,
^{\href{#endnotes}{10}} and many online news sites have a viral video section themselves.
It was, therefore, surprising that this type of content did not feature in our
sample period.^{\href{#endnotes}{11}}
While the Web can't be directly compared to television, because only five
days of content were analyzed, when we drilled down to the specific stories
covered using UGC on television during the sampling frame, similar patterns
are visible.
TABLE 5: The Specific Stories Covered Using Some Form of UGC

Of the five stories with the most UGC on television (Syria, the Glasgow
helicopter crash, Ukraine protests, Egypt protests, and Black Friday), four of
the top five stories on the Web were the same, the exception being Black Friday
coverage, which did not feature as prominently online. This is because
Black Friday fell on November 29, which was not one of the five randomly
sampled dates. Instead, the protests in Bangkok, Thailand, received coverage
on four out of the five sampled days.

But these pure numbers alone do not accurately reflect what was happening
over the three-week period. When stories are mapped against date,
the resulting graph (FIGURE 3) shows the three clearest patterns from the
research.
FIGURE 3: Comparison of the Amount of UGC Used on Television Over the 21-day Period

FIGURE 4: The Top Stories Including UGC on the Web

Our interviews highlighted significant differences in the use of UGC
between national broadcasters and 24-hour news channels. National bulletins
are far less likely to use UGC. Many journalists working at national
news organizations admitted to relying entirely on the agencies for content
on international stories such as Syria, but did suggest they would ask their
audiences to send in content during national stories. At this national level,
there remained a concern about seeking out content from social networks,
due to verification issues. Editors seemed more likely to trust content
sent directly to the newsroom than content sourced from the social Web.
Gudrun Gutt from ORF in Austria explained:
One year ago we had flooding in Austria and we said, ``We really
have to start gathering content from the people in the flooded areas,
because we couldn't even reach them with our crews.'' So we [put out
calls to action] on Facebook, and during our news we put out messages
and we asked them to send us our content. That is the type of
UGC we gather. The type of UGC that we use mostly is the [material]
which is delivered by Eurovision [News Exchange] and validated
already. What we don't have yet is a real, dedicated desk that validates
UGC— let's say, from Syria.
\section{When is UGC Used?}
\begin{enumerate}
1. UGC was used to tell the story of the Syrian conflict almost
every day.
Thirty-five percent of all the UGC analyzed as part of this research related
to the Syria conflict (40 percent of the UGC used on television, and 20 percent
used online). This content appeared almost every day during the sample
period on at least one of the channels under analysis. This is reflected
in the consistent presence of the green column in FIGURE 3 and the navy
blue column in FIGURE 4. Of the 2,115 times UGC items coded as appearing on television over the three weeks, 842 concerned Syria. However, those
842 pieces were broadcast over the entire period. By contrast, all 349 of
the UGC items identified during the Glasgow helicopter crash appeared on
November 29 or November 30, and December 2.
Covering the Syrian conflict has been an ongoing challenge for news editors.
Limitations placed on foreign journalists to enter or move freely within
the country have meant news organizations' reliance upon UGC as a way of
telling the story, and our interviews underlined this dependence. As Reuters
admitted, ``The activist videos have really formed a foundation of the reporting
that comes out of that story.'' The AP provided a similar answer, saying,
``We don't use UGC as a replacement. We do send people into Syria when
it's safe to do so but UGC is the way that we've been able to tell the story.''
An important point some people raised during interviews was that, whereas
UGC during breaking news events often produces the most dramatic pictures,
in coverage of the Syrian conflict sometimes a package included up
to 10 six-second clips from 10 different YouTube channels of white smoke
rising against a blue sky. One producer admitted honestly:
I wonder—when all [the audience is] seeing is continuous Syria—if
you could almost run the same picture every day and would anyone
notice? That's what worries me. Lots of footage of exteriors and the
only way we can tell the story is by using these pictures. I'm telling
you I'm [putting them] out on air, and I'm thinking this is boring.
And I shouldn't say that because people are dying. I think there were
really significant videos with the chemical attacks and the barrel
bombs. That introduced us to a new style of warfare that we hadn't
seen before. That was important. It has its moments, but then it's
same, same, same, puff of smoke, puff of smoke. I just wonder about
viewer fatigue on that sort of thing.

Certainly Syria could be considered as an outlier. However, we started this
research expecting almost all uses of UGC to be around Syria. We were
actually surprised that Syria footage didn't comprise more than 35 percent
of the UGC analyzed. The main reliance on UGC was around breaking
news events.
2. During breaking news stories, UGC fills the gap while news
organizations wait for other pictures.
The yellow column in FIGURE 3 illustrates the amount of UGC used in
the coverage of the Glasgow helicopter crash. The crash occurred late in
the evening (GMT) of November 29, causing a peak in UGC on November
30, when pictures first emerged. Of all the one-off stories (i.e., not ongoing
stories like Syria, Ukraine, and the Thailand protests), coverage of the helicopter
crash included the most UGC use.
There was a clear peak on November 30, because, in the first hours after the
crash, most news organizations relied on pictures taken by eyewitnesses
and posted on Twitter. For example, BBC World broke the story at 11:08
p.m. GMT, and over the following three hours it used 35 minutes and 15
seconds of UGC. Those 35 minutes were made up of four pictures sourced
from Twitter and an unidentified 13-second video of a police cordon. While
the economic element of UGC is not part of this phase of research, it has to
be acknowledged that 35 minutes of free content is a significant amount of
money for a television news channel to save.
The crash happened late on a Friday night in Europe, a time when newsrooms
are traditionally lightly staffed. However, as the story developed,
news agencies and the news organizations themselves were able to get their
crews in place in Glasgow. When professional images started to come in,
the reliance on UGC was noticeably reduced.

During our interviews, journalists at larger newsrooms talked about sourcing
pictures they had discovered themselves on the social Web as a way to
fill the gap before agency pictures arrived. One journalist talked about covering
the Kiev protests. He explained:
[When the Lenin statue was knocked down] I knew professional
photographers were there, but I could not see those pictures on the
wire yet. I don't know how they work, but those 10, 15, 20 minutes
it took before the pictures showed up on the normal wire we had
used a picture from this guy from Kiev, [which he had] posted on
social media.
3. UGC is used when no other pictures exist.
As the previous section illustrated, UGC fills a gap before other pictures
emerge. It also drives stories that otherwise wouldn't be told. During our
sampled time frame, secretly filmed UGC exposed serious police brutality
in Egypt and the Ukraine, footage captured from a dive rescue team's cameras
showed the unexpected discovery of a man alive in a sunken ship, and a
group of children in Damascus narrowly avoided a mortar shell that landed
near to where others were talking to a camera in the street.
Someone who works on a foreign desk conceded that ``[UGC] makes it possible
to tell stories that you wouldn't have previously told because of lack
of pictures.'' Indeed, an interviewee based in the Investigations Unit at a
San Francisco television station described how secretly filmed videos play
an increasingly prominent role in tipping off journalists about the need
to undertake investigations into certain subjects. As people don't tend to
upload that type of content to the social Web, he emphasized the need to
build strong relationships with the audience in order to encourage them to
alert news outlets to these types of stories.

Similarly, we saw a piece of drone journalism used by multiple broadcasters
during our sample period. A citizen journalist in Bangkok took the
footage during the protests of November/December 2013. As a research
team, we had noted the absence of any UGC from the Bangkok protests and
concluded this was likely due to the city's status as an international media
hub—broadcasters either had their own camera people there or relied on
the agencies to provide enough content to fill the one to two minutes dedicated
to this story every day, we suspected. However, when the drone footage
emerged, the aerial shots were so powerful and entirely distinct from
the pictures coming from the ground that a number of the broadcasters
in our study ran them. Drone footage is appearing much more frequently,
particularly during large-scale protests.^{\href{#endnotes}{12}} As a side note, Scott Pham, who
undertook a research study into drone journalism, explained:
In some ways UGC is the best way to get drone photography because
of the legal situation around creating it yourself. A lot of organizations
are really seeking that kind of content out. In some ways that
might be driving drone journalism in an era where doing it professionally
is very difficult [because of the regulations that currently
exist]. In some ways the amateur drone journalism going on might
be more interesting.
4. UGC was used as part of news programs dedicated to the Web.
One trend that has emerged over the last couple of years is 24-hour news
channels producing programs or segments entirely dedicated to the Web
and social media. On Al Jazeera English, it is called The Stream, on BBC
World, it is BBC Trending, on France 24 there are two: Les Observateurs and
Sur Le Net. These programs tend to be 15 to 30 minutes long, and focus on
those topics trending on social media before launching into longer pieces
of journalism around the subjects. What's notable here is that, where elsewhere
particular channels might seemingly try to hide the fact that they
were using UGC—by either failing to label or credit appropriately—these

dedicated programs went out of their way to emphasize their use of it. So,
for example, rather than taking down clips from YouTube and using them in
packages, one show played a YouTube video using a computer screen so the
content's source was quite clear. There was such an emphasis on the social
networks that, during an episode of The Stream about Nelson Mandela's
death, the original footage of him leaving Robben Island prison was played
from YouTube, even though the event itself was over 20 years old.
Elsewhere, France 24's Les Observateurs, a program entirely dedicated to
UGC, focused on uploaders, giving them a platform to tell the story of
the event they had captured from their own perspective. The uploaders
appeared on screen, and were named in the final credits of the program as
if they were producers.
\section{Why is UGC Used?}
There were eight different reasons given for why UGC is used; some practical,
some about quality, some editorial, and some relating to the audience.
The most frequent reason given for using UGC was that it provided the
only available pictures. As one journalist argued, ``After an event, all you can
take a picture or film of is the police blue line.'' Another editor talked about
receiving footage of a flood. ``It was absolutely in the moment. The footage
was actually shot through a car windscreen and the windscreen wipers are
going crazy. It had quite good audio on it as well. So, you got the real drama
of the storm, not just the damage that was left behind.''
Speed was also mentioned. ``UGC is so much faster. It's ridiculously fast,''
said one interviewee. Another online journalist explained, ``In the online
business it's very competitive between news sites and we can't really afford
to wait 30 minutes for a picture on the wire. So if social media can give us a
picture instantly, we will use it.''

Other journalists talked about specific characteristics of UGC. ``In places
like Syria or Egypt, or even in Ukraine, [people with phones] become your
eyes and ears on the ground, and they're able to feed content to you from
up close. [The footage] can be a lot more personal than maybe [professional
journalists] would [get] carrying a big camera with them.'' Another editor
echoed this point, saying, ``It makes it feel real because rather than having
someone standing in front of a camera—you know, your average white bloke
in a tie—you have something handheld and jerky… It makes it feel more
real and gritty.'' Chris Hamilton, from the BBC, cited the London bombings,
saying there were lengthy discussions about whether the audience would
accept the shaky footage taken by people being led to safety through the
underground tunnels, compared to now, when editors are specifically looking
for unsteady UGC footage because they know the audience equates that
with authenticity. Now, editors can worry that footage filmed by people on
the ground with their HD camera phones looks too slick.
Another benefit of UGC raised by some journalists was the diversity of
voices it provides, extending beyond traditional sources listed in internal
contact databases. One editor cited Ukraine as an instance where UGC provided
``different angles and views that we were not getting from anywhere
else.'' Another senior journalist explained how UGC was integrated into
coverage of the same story:
We get a lot from Ukraine and it gave us diversity. We would have
been able to cover it [anyway] because we ultimately put two or three
teams there in Ukraine—a couple of correspondents and representatives—
but we really did use UGC for a diversity of voices on the
ground. This is especially the case if you're a major TV broadcaster
and your correspondents are tied to the shot 24-7. They're in a live
position. We try and let them off to do their newsgathering, but most
of the time they can't leave that satellite position.

One BBC journalist explained how social media has changed its newsgathering
techniques. ``A couple of times I've been really, really stuck and I've
thought, ‘I'll have a look in the World Service contacts [directory],' but I feel
like a failure if I do that. I think it's right that it feels like that because you've
got to have new voices on air.''
Other journalists explained how UGC has allowed them to continue the
life of stories after the rest of the mainstream media had moved on to other
events. David Doyle from Channel 4 News in the United Kingdom outlined
his experience covering barrel bombings in Syria, saying, ``An example of
one that we've done recently was barrel bombings. It's been going on a long
time so it stops being a story, but by using the user-generated content, we
were able to get some very striking images. It allowed us to explore this
phenomenon in depth. [The UGC] really brought home what is happening
on the ground.''
UGC helped Channel 4 keep what Doyle called a ``war crime''^{\href{#endnotes}{13}} in the
public eye:
Basically, it's this huge human rights violation. It's a war crime that's
going on but because it's sporadically happening across the country
the victims are often smaller in numbers—smaller numbers than
there were in the chemical attack in August. Therefore, it doesn't get
picked up in the same way, but you can comment and really bring
home what is happening on the ground.
While our study did not include any element of audience research, some
journalists talked about the ways in which they feel UGC strengthens their
relationship with viewers, providing them with the opportunity to become
part of the storytelling process. One European editor argued, ``I think [UGC]
deepens our relationship with the audience.'' Another journalist, who works
daily with communities creating content, spoke passionately about the way
UGC allows people to tell their own stories. Using protests in Istanbul's

Gezi Park during May and June of 2013 as an example, she described how
the protestors told their own stories. ``We didn't tell their story for them and
I think that's very, very important,'' she said.
\section{UGC as a Substitute?}
Despite the range of reasons provided for actively integrating UGC into
output, one point that remained prominent throughout our interviews was
that UGC does not, and cannot, constitute a replacement for professional
journalism. A digital editor underlined this point, saying, ``We mustn't overstate
the importance of UGC... It's incredibly important in some instances
where professional organizations can't get [somewhere] at speed, but it
doesn't supplant much. We mustn't see it as better than the other.''
Certainly a couple of editors were quite adamant that they only used UGC
out of necessity. One argued:
We're only using this stuff because we're not there. We're using this
stuff because we can't get to these places anymore, whether it's Egypt,
Libya, Syria, Iran, as well as other countries that are not necessarily
unsafe, but you can't go because you can't get visas. I mean, if it was
easier to film in these places we wouldn't be using social media; we'd
be filming there ourselves.
Another editor, from the other side of the world, agreed. ``It's good for breaking
news and it's good for a first response, but I still—every time—would
prefer to have my people in the field.''
\section{Conclusions}
User-generated content is used when other images are not available, as the
ongoing reliance on it (even by national news bulletins) to cover the Syrian
conflict demonstrates. The way that UGC was integrated during coverage of

the Glasgow helicopter crash and the razing of Lenin's statue in Kiev during
the Ukrainian protests suggests that it is often employed as a stopgap before
news agency pictures emerge—interestingly, even if the professional ones
are less dramatic. During interviews, even though some journalists spoke
passionately about the benefits UGC provides in terms of authenticity, a
diversity of voices, and news angles, there was a recurring argument that
where possible, professional journalists should be telling the stories.
UGC certainly inspires stories that would otherwise be ignored, as long as
the pictures are sufficiently compelling. Within our sample UGC emerged
to drive a handful of stories. Some were kicker stories like the one mentioned
previously about a ship's cook who was unexpectedly found alive by
a dive team sent to investigate a sunken ship. Others were shocking cases
of police brutality captured through secret filming on camera phones. As
newsrooms become more confident in discovering and verifying content, as
well as building relationships with different communities, the opportunity
to use UGC to report on previously ignored stories will grow.

\chapter{Workflow}
As the amount of UGC produced around news events has proliferated,
newsrooms have had to discover, verify, and secure rights for using this
content. They have also had to find ways to make it broadcastable by
transcoding video files, while developing systems for distributing it across
news desks.
There is no one established workflow in terms of the way UGC is treated
across the news industry. Just considering the issue of discovery, there are
four methods newsrooms use for finding and accessing user-generated content:
a) locating people with footage at the scene of a breaking news event,
b) encouraging people to share photos or videos directly with the newsroom,
c) finding content on social networks themselves, or d) relying on
agencies or third parties once the necessary rights permissions are in place.
In most cases, smaller, national newsrooms rely solely on the news agencies
for international stories. On domestic stories they may, however, encourage
their audiences to submit content, or turn to Twitter or Facebook to search
for photos, videos, and eyewitnesses.
It is relatively rare for a newsroom to have a dedicated UGC desk, as the
BBC has with its UGC Hub. Still, even in news organizations where responsibility
for discovering and verifying UGC is shared across the newsroom,
there is normally a team of people (or even just one person) seen as having

more expertise in handling UGC. These journalists are much more likely
to have Arabic language skills given the volume of content emerging from
Syria and Egypt.
Due to the limited number of people who have the necessary skills to verify
UGC, and the concerns many senior editors share about the difficulties of
verifying this type of content, the final decision to use a piece of user-generated
content often has to be ``signed off on.'' This means that the time of
day often impacts UGC use. During overnight shifts and weekends, UGC is
much less likely to be used.
The biggest challenge that larger newsroom face involves safely sharing
content around a newsroom once it's been discovered and verified. Even
downloading video from YouTube and converting it into a format that can
be broadcast provides a real headache for many newsrooms. One newsroom
has even set up a camera to film a computer screen, which plays
the YouTube videos they want to use. Once the footage is converted and
uploaded into internal Media Asset Management (MAM) systems, information
related to the completed verification checks and required crediting
can be, and is often, lost. As a result, by the time output or gallery producers
receive the footage, they often don't know the pictures' origins or that they
are working with UGC at all.
\section{How Newsrooms Discover UGC}
The primary method of UGC discovery differs by newsroom. Key variables
for this are newsroom size and geographic reach. National news organizations
tend to rely almost entirely on agencies for UGC to complement their
international stories. They have neither the audience reach to expect eyewitnesses
to send them content directly, nor the internal resources or expertise
to scour social networks for reliable and trustworthy content. These
organizations do, however, tend to look actively for UGC to supplement
their coverage during domestic news events, such as bad weather stories,

riots, and elections. As one high-level news manager explained, ``We get
UGC mostly from news services. Stuff that comes in independently would
be more domestic than international.''
\begin{enumerate}
1. Content is located at the scene of a breaking news event.
A number of interviewees referenced the piece of footage that emerged
around the murder of Lee Rigby on the streets of Woolwich in London
in May of 2013. The footage, which features one of Rigby's killers talking
directly into the mobile phone of a passerby, explaining his actions, was
considered a watershed moment for UGC. Purchased by ITN for an undisclosed
fee, the video was used by news organizations across the world either
via syndication, partnership distribution agreements, or fair use. Many of
the UK broadcasters with whom we spoke discussed how this event triggered
internal discussions about buying footage from the scene of breaking
news events. In particular, newsrooms identified a need to send senior journalists
with the authority to spend money, as well as ensure that producers
had the appropriate contracts ready for people to sign at the scene. One
senior broadcaster shared his thoughts about the future of newsgathering,
arguing that the best pictures will always come from eyewitnesses on the
ground who have captured footage on mobile phones. He wondered aloud
whether a smarter use of resources would be to send producers out to purchase
exclusive content, rather than sending their own cameras to film.
2. Newsrooms encourage people to send photos or videos directly.
Large 24-hour news channels have audience-reach, which means they are
often sent pictures and videos directly. The BBC, in particular, still has the
luxury of a global audience, many of whom know the yourpics@bbc.co.uk
email address. CNN has iReport, a citizen journalism project established in
2006, which has built a very active community of ``iReporters'' who respond
to daily calls to action, some of them linked to softer features' topics, but
many connected to hard news events. In addition, Al Jazeera Arabic has its

own very successful UGC portal called Sharek. According to research carried
out by Juliette Harkin and colleagues, at the beginning of the Syrian
conflict, the Sharek portal was receiving ``more than 200 videos per day and
up to 1,000 videos on Fridays.''^{\href{#endnotes}{14}} Its success led to the launch of the organization's
Mubasher channel. Mubasher, the Arabic word for live, features
continuous live streams, many of which are filmed by citizen journalists on
the ground. During our coding of Al Jazeera Arabic, a large number of live
streams were used, some of which ran for minutes at a time. Our analysis
also showed that Al Jazeera Arabic used an average of 50 pieces of UGC per
day, considerably more than the 11 pieces averaged by the other channels.
The success of the Sharek portal has clearly played a significant role in the
amount of UGC used on a daily basis.
Newsrooms prefer to receive content directly, as it provides them with
exclusive content and the terms and conditions outlined on their websites
mean that contributors have already accepted that they are giving the newsroom
certain rights to use the content across the organization and its partners.
However, because audiences have become used to sharing the content
they capture on their own social networks, many interviewees talked about
the challenge of encouraging people to send them content directly. People
who work on successful UGC initiatives such as iReport, and the more
recent GuardianWitness project, talked passionately about the need to
build relationships with the audience. They emphasized the need to think
about content creators as a community, showing contributors how their
content had been used and how it had improved the news organization's
storytelling. They were adamant that this was the key over sitting back, simply
thinking that a general call to action after a news event would result in
high-quality submissions.

3. Newsrooms find content shared on social networks.
Many interviewees discussed how their newsrooms have shifted from dismissing
the notion that social networks can be used as a newsgathering tool,
to becoming increasingly reliant on them. As one digital editor explained:
Everyone has got Twitter, Tweetdeck, or Hootsuite open on their
desks. Everyone has. Even those who used to be quite reluctant to
engage with it, and thought it was all about Miley Cyrus, they are all
now doing it. It's good for us, because rather than being the people
who have everything open and are expected to find [UGC], we're the
ones who say, ``Be careful with that.''
Another editor talked about the way that social newsgathering has been
integrated within the newsroom. ``It's totally embedded. Our people on the
news desk are monitoring social media all the time for tip-offs, pictures,
videos, sources.''
A small number of newsrooms admitted rarely using UGC within their
output. One example shared by RUV, the national broadcaster in Iceland,
highlighted the differences that can still exist between international and
national organizations. In December of 2013, Iceland experienced its firstever
instance of a police officer shooting and killing a civilian. CNN, thanks
to someone in its iReporter community, received UGC coverage of the
immediate aftermath. RUV did not use any UGC in its reporting of this
major domestic event. Searching for UGC is not currently part of RUV's
newsroom mindset and it does not have an equivalent to CNN's community
of iReporters.
Although our research was focused on how newsrooms find content after
a news event occurs, our interviews and observations revealed that Twitter
is a primary means by which reporters are alerted to breaking news. CNN
is currently using DataMinr for a trial, and journalists certainly felt that
the alerts were consistently quicker that the traditional news agency wires.

Similarly, the BBC has a position on its central newsgathering hub called
the ``Live and Social'' position. There, one person is charged with monitoring
Tweetdeck, and is fully loaded with lists of verified sources to give the
BBC a leg up on breaking stories. Other newsrooms rely on the @Storyful-
Pro Twitter account that pushes out verified breaking news alerts sourced
from Twitter.
4. News agencies play a role.
One of the main findings from this research is the role news agencies play
in the workflow that surrounds UGC in all broadcast newsrooms.^{\href{#endnotes}{15}} For all
organizations we interviewed there is a heavy, if understated, reliance on
the main television news agencies AP and Reuters. All the television-based
organizations we interviewed were subscribers to one or both of these agencies.
There are other smaller agencies competing in this space, such as AFP
TV; the television arm of the French news agency Agence France Presse;
and Ruptly, the agency arm of the Russian broadcaster RT.
Storyful,^{\href{#endnotes}{16}} a social media news agency which only discovers and verifies
UGC, has acquired a significant client base in a short period. Also important
are content exchange platforms, such as the Eurovision News Exchange
operated by the European Broadcasting Union (EBU). This platform primarily
enables the members of the EBU—Europe's public broadcasters—to
exchange content between each other.^{\href{#endnotes}{17}} It also allows Reuters TV and AP to
distribute content, including UGC. The Eurovision News Exchange is also
a client of Storyful on behalf of its members and distributes content discovered,
verified, and cleared through them. Many of the public broadcasters
interviewed acknowledged the role that the Eurovision News Exchange
plays in distributing UGC for their use. Fifty percent of the news organizations
coded in the quantitative part of the research were Eurovision News
Exchange partners.

For almost every newsroom, the news agencies play a very significant role in
terms of discovery, verification, and rights clearance. Surprisingly, the scale
of their impact is not always recognized, with some smaller national newsrooms
actually stating they ``don't use UGC,'' without realizing that many
of the pictures they accept from the agencies are actually user-generated
content. (This is particularly troubling as UGC distributed by the agencies
is very clearly labeled as such, perhaps illustrating just how few journalists
read dopesheets properly!)
For other newsrooms, the role agencies play is fully appreciated. As one
social media editor explained, ``My personal take is that if something happens
we would be one of lots of news organizations wading in, going, ‘Can
we use your picture?' whereas actually that is what Storyful does for us.''
As well as simply integrating content that the agencies distribute, some
newsrooms that discover content themselves will actually send it to the
agency. ``We do get clients flagging stuff up to us, [saying], ‘Have you seen
this?'… We want to be belt and braces certain of something before we use
it, [which is why certain newsrooms will] give [UGC] to one of the agencies
and see what they can do with it.''
These platforms have always been an important source of international
news coverage. The news agencies are often the first into conflict zones or at
breaking news events and the last to leave—and broadcasters have relied on
this for years. Nothing has changed in this regard when it comes to UGC.
The agencies' clients or partners demand it, so much so that they have had
to rebalance and learn how to source and verify UGC to the required standards.
One foreign editor at a medium-sized channel clearly explained the
benefit for the broadcasters, saying, ``Arguably, that's the best way of getting
UGC. Somebody else has done the work for you, because that's what we pay
them for.''

We conducted interviews at both the AP and Reuters, and because of their
impact on the use of UGC within newsrooms around the world, it's important
to examine exactly how these agencies work in terms of UGC. We
requested an interview with AFP, but received no reply.
AP and Reuters have taken slightly different approaches in their UGC workflows.
The AP has established a dedicated social media desk, with a social
media editor and producer. Reuters TV has kept the role as part of the main
news desk's responsibilities. What was noticeable about the workflow at
both agencies was the reliance on bureaus and staff on the ground. As Fergus
Bell from the AP explained:
Through training we have got the whole staff onboard so everyone
knows that they have to monitor social media. If it's video then people
know that they can send it to me and I can take on the verification
or advise them on the verification. But text, video, and photojournalists
at the AP have all been told that they need to be aware of things
on their patch, that if there's verification needed for something in
their patch then they will be the ones having to do the verification.
I've seen so much that I can usually spot something that's not right a
mile off, so we put it through as many eyes as possible. Any contentious
UGC gets alerted to senior managers, to senior editorial management
before it gets put out. So there's lots of eyes to catch it, but
also the responsibility is on the experts in the region.
Reuters relies similarly on its own network of bureaus, but does not have
a dedicated verification desk at its London headquarters. The decision to
use UGC rests with the editor of the day. As Soheil Afdjei, news editor for
Europe, the Middle East, and Africa, explained:
Our structure for UGC is developing as we use more of it. The way
we're set up in terms of the number of bureaus we have around the
world gives us that foundation to search for UGC around an event at
a bureau on a regional hub level, where the story has happened and

people are closer to the story. They know the territory better, they
speak the language, they can cross-reference with colleagues who
work for us on the text side or the pictures side [of the house]. In
terms of using it and publishing it we then go through the verification
process. It's verified at the bureau level, regional hub level, then
it comes to us and the editor of the day. They look at it and we discuss
the merits and the caveats for running it, or not running it.
That said, both agencies were clear that at no point do they think that UGC
replaces their own work. As the head of output at one explained, ``UGC is
an extra tool in the box, and it supplements the work that we're doing, but
at no point does it ever replace it.''
\section{Staffing Around UGC}
The question of whether all journalists in a newsroom are responsible for
monitoring breaking news on social networks and finding videos and pictures,
or whether it should be given to a small unit or even single journalist,
has not been resolved.
There was certainly a sense that all journalists should have an understanding
of how to discover and verify content, but there was also a belief that
having dedicated staff that work with UGC every day is preferable, because
those employees are better able to develop and become specialized in the
techniques required.
As one editor observed, ``My feeling is I don't think the guys who are at the
sharp end [the duty editors] are ever going to be in a position to be properly
across social media, and I think we do need a dedicated role doing that.''
CNN talked about the ways social newsgathering has been integrated
across the newsroom, highlighting the role of its iReport staff as clearly
very important.

[Social newsgathering] is totally embedded. Our people on the
news desk are monitoring social media all the time for tip-offs, pictures,
videos, sources. We also have the iReport, so every morning
in our editorial [meeting] we have an iReport representative at the
table and they may say, ``Oh, we're getting really fabulous iReports.''
… They can be stills, video, blogs, so it's totally integrated for us in
two ways. We can call out to our iReporters because we've got thousands
of them now. We also monitor social media for breaking news
[using DataMinr].
The biggest dedicated UGC unit within the news industry is the BBC's UGC
Hub. The desk is staffed by around 20 journalists covering the whole day in
shifts, with a peak staffing level of five journalists during the day working on
regional, national, and international stories across all BBC output—television,
radio, and online. This desk is based in the BBC's main newsroom, and
one of its team members sits on the central desk that coordinates all the
BBC's news output, while the remainder of the team sits at the edges of the
main news area.
ARD and ZDF in Germany have content centers that work with UGC alongside
other content intake tasks. At ZDF, the tasks associated with UGC were
added to the responsibilities of a pre-existing team. ZDF's editor-in-chief,
Elmar Thevessen, explained how doing this changed the team's importance
in the newsroom and made them more visible. It put UGC discovery into a
more central place:
They will either search the Internet for video material themselves, or
they will get a demand from one of the different programs that we
have that they should look for specific material. And they are now
connected into our news desk, which is something we hadn't done
before. In the past, they were operating totally separately from everybody else, but now we basically put one of them right in the middle
of the newsroom; we also built a little office there. They work in shifts
from 5 a.m. in the morning until 1 a.m. the next day.
Other, smaller organizations with daily news bulletins have no dedicated
desks, but have allocated the role of UGC or social media editor to a single
staff member (in many instances, due to the Syria conflict, this person is an
Arabic speaker, as at VRT of Belgium and Channel 4 in the United Kingdom).
Having one person in-house ensures that workflows surrounding UGC are
taken more seriously within an organization. However, having only one
specialized UGC staff member also causes problems when they are not
on shift. The pan-European news channel euronews has a single dedicated
UGC editor, who noted, ``When I'm not working, usually the rule is not to
use UGC. It's the best way, but that means we can be a little bit limited.''
This can also mean that when UGC is used outside of that window mistakes
can happen, as other producers are unaware of the correct procedures. This
was the explanation euronews offered for its limited crediting of the amateur
content it put on air during the Glasgow helicopter crash. Notably, it
was not only that the UGC editor was not working; general staffing was
also lower. Peter Barabas, the channel's editor-in-chief, explained, ``Staffing
is reduced by 40 percent on nights and weekends. We function at about 60
percent of what we do on weekdays.'' A social media editor from another
channel noted that when he was out of the office UGC that comes from
agencies often doesn't get credited. He explained that because his organization
doesn't ``credit pictures that come from agencies, because we work
a lot with agencies,'' people thereby think the same rules apply to agencygenerated
UGC.
These single-editor organizations are clearly aware of these risks, but believe
that having a dedicated UGC specialist ultimately provides more benefits.
As the senior foreign journalist at VRT confirmed, ``You have to remember
that we are a small organization and in bigger organizations it's probably a
bit more organized. When you sometimes see them on the BBC talking to

their Syria desk, that's impressive. For us it was already a big thing that we
have a producer who looks [at content from] the Arab world.'' It was clear
from our interviews that the benefit of having at least one staff member
responsible for UGC means it provides a model for UGC workflows among
other producers in the newsroom.
Organizations without staff dedicated to UGC-discovery typically point to
their size and/or budget. Interestingly, however, they recognize that this
will need to be rectified in the near future. As one editor admitted, ``It's
something that is playing an increasingly important role in what we do. It
is something that its time will come. We have a digital desk that generates
digital content and it would seem to be a natural extension to their duties.
Not that they'll be thrilled to hear that.''
Certainly, where there is no dedicated UGC or social media desk, the online
teams tend to be tasked to handle user-generated content. As one digital
editor said, ``Let's say a building collapses or a fire happens. The TV people
will come over to the online team saying, ‘Can you see if there's any footage
out there? Or any photos we can use?' '' But as another digital editor conceded,
``It used to be that [when a story broke] the online team would pop
up like meerkats saying, Look there's a photo! Now everyone in the newsroom
[knows how to find that photo].''
\section{Conclusions}
There are four ways that UGC finds its way into news output. For some
organizations, the only way it arrives is as part of a news agency feed. It has
been verified, cleared, and is in a format that can be dropped straight into a
package. Other news organizations scour social networks for UGC during
breaking news events. But that requires a great deal more work in terms of
verification and rights clearance, and with YouTube content that is to be
broadcast on air, there are the additional issues of downloading and converting
the video to a suitable format.

When newsrooms are using their own newsgathering techniques to source
and verify UGC, be it via social networks or by people directly sharing
material with them, there are staffing implications. In smaller newsrooms,
one person might do the job; in others there are dedicated desks. Certainly,
staff that work with UGC every day believe a dedicated desk would make a
significant difference to the newsroom.
It was acknowledged that the basic skills required to work with UGC have
developed across newsrooms, but there is a recognition that people who
do this every day are able to hone their skills and expertise in very important
ways. At the BBC, which has its dedicated UGC Hub, Chris Hamilton,
the social media editor, explained that while many people in the newsrooms
know how to spot fake Twitter accounts, that skill isn't universal.
He explained, ‘[The Hub] is still a go to, certainly for anything that is truly
amateur, and especially on Syria.” He also discussed how skills are being
disseminated across different desks. ``There are pockets of expertise [across
the newsroom], partly because people work on the Hub and then they go
off and work elsewhere.''
Our interviewees demonstrated that newsrooms globally are using more
UGC, and even those who currently use very little acknowledged this is
going to change. As this occurs, the way that UGC is managed, and by
whom, is going to be an increasingly important question for senior editors.

\chapter{Verification}
The theme of verification ran through all 64 interviews. Managers and
senior editors were quick to emphasize the importance of verification in
relation to user-generated content, as well as their fears about using material
that turned out to be incorrect. Many people discussed famous examples
of news organizations being faked by hoaxes, and their concerns about
it happening to them.^{\href{#endnotes}{18}} As one senior news manager stated, ``I think the biggest
issue for us is around verification because that's where our reputation
lies. If we end up putting stuff out which is wrong in any way, fabricated in
any way, then our necks are on the line.''
There was, however, little awareness about the specific techniques and processes
associated with verifying UGC.^{\href{#endnotes}{19}} People knew it needed to be done
but there was an acknowledgement from journalists on news desks that
they didn't feel like they knew enough about how to do verification properly.
One very honest description of an editorial meeting by a senior editor
revealed how the process of verification is considered in that newsroom:
Verification is always an afterthought. It's sort of like, ``Let's just get
it on air and online and then not worry about it.'' It's always an afterthought.
When someone puts something forward in an editorial
meeting, you say: ``Have you verified it?'' And people groan. People
are scared of the ``v'' word. They know it's going to take a long time.

There was still a sense from many managers and senior editors (who don't
work with UGC every day) that with journalistic experience comes a gut
instinct that enables you to know whether something can be trusted. There
was also a sense that verifying a piece of content is something that is black
and white—something is either true or false, accurate or inaccurate.
When pushed to describe specific technical checks that journalists could
run on social content, very few interviewees displayed any knowledge of the
different ways geo-positioning and timestamps work on the different social
networks, the power of mapping technology, or the information about a
digital photograph available via EXIF data.
It was rare to hear people talk about verification as a process, like building
a legal case. They should be looking for clues that help build that case, and
in almost all pieces of UGC there will be some doubt about one element of
the material. Whether or not to run UGC rests then with an editor of a program,
section, or article. The fact that the process of verification includes so
many different factors and variables means these decisions are very rarely a
simple case of true or false.
Many people who didn't regularly work with UGC described social media
verification as having the same characteristics as any other type of factchecking;
those who work in roles where social media verification is part of
the job, however, talked about it very differently.
Verification Processes
The AP has a clear process whereby the uploader of the content has to be
verified separately from the events being shown in the footage. Similarly,
Storyful verifies the source, date, and location of each video separately—
labeling each element as either confirmed, corroborated, or unconfirmed.
This information is then shared with clients. Information about the checks
carried out by these two agencies are detailed on dopesheets, allowing their
clients to undertake their own verification checks if they so wish. Some

newsrooms carry out an additional layer of independent verification on
material shared by the agencies, but the vast majority do not, believing that
``is what they are paying agencies for.''
Perhaps most alarming was the ignorance about the problem of scraping,
the practice whereby people simply copy pictures and videos and upload
them to their own accounts. As someone from an agency warned:
A fundamental problem that the entire industry faces is that usergenerated
content is used in its most available form as opposed to
its original form. It is quite likely legitimate in the sense that it shows
what happened, but I think that's a major problem because it takes
away all of the context and all of the original information that is connected
to the video. I think it's because people use technology to
surface what's essentially trending as opposed to finding the original
piece of content or tunneling through to see what the original
posting was. If there isn't any original information and context, then
what's been added to it in the version you're looking at may or may
not be true.
Even analyzing the tweets and messages journalists send to people who
have uploaded footage to the social Web immediately after a breaking news
event demonstrates how rarely journalists think about these dangers. They
will ask for permission to use the picture, without asking whether the actual
person took the photo or shot the video. This clearly has issues related to
copyright, but it has even bigger issues related to verification.
The question of checklists and systemized processes was asked of every
interviewee. There was resistance about the need for standardized verification
systems, with people arguing that every piece of content is different
and on desks where UGC is regularly used, there was an acceptance that
staff just knew which checks had to be completed.

However, the people who are making decisions about output displayed
the most ignorance about the technical checks that could be run, and how
these could be integrated with traditional verification and fact-checking
techniques. They were the ones who were most likely to rely on a ``gut
instinct.'' More UGC-savvy producers suggested the need for implementing
clear flagging or traffic light systems whereby pressured output editors
could quickly see which checks had been run, which elements had been
confirmed, and which elements had been corroborated but not confirmed
to visually represent the reality of the sliding scale of verification.
\section{Who Should do Verification?}
The question of who should do verification differed from newsroom to
newsroom. The model of the BBC's UGC Hub has not been replicated
on a similar scale, although there were certainly newsrooms that realized
the importance of creating special desks, even if they only contained one
person specifically working with UGC. These special desks were predominantly
focused on content coming from Syria, and increasingly Egypt, and
were frequently staffed by Arabic speakers. But again the emphasis placed
on these people was about their knowledge of the location and language,
and therefore their ability to discover original content and cross-reference
this with local expertise. It was very rare that staff on these desks had been
given specific training on verifying online content.
Overall, there was a sense that all journalists should be responsible for verifying
the content that they discover, in spite of very little specific training
on verification (with some exceptions, notably ABC Australia) of content
discovered on the social Web or sent in to newsrooms via email or upload
technology. There was only one senior manager who discussed the need for
training people on how to make both editorial and technical assessments of
content sourced from the Internet.

And while there was agreement that skills have improved slightly—for
example, more people know how to do a Google Reverse Image search or
know how to identify fake Twitter accounts—there was also an acknowledgement
that there is quite a lot of ignorance about ways to verify content
systematically. As always with digital skills, it is impossible to know what
you don't know, and many people do not know what is possible, whether it's
the possibility of using Wolfram Alpha to check weather reports for a certain
location on a specific day, the ability to check who registered a website
or blog, or the information included in EXIF data. There did appear to be
a trend among television journalists who often send material they discover
to the online team, believing that people working in the online space have a
better sense of whether something can be trusted.
\section{Reliance on News Agencies}
There is a significant reliance on the agencies for verification, and the majority
of newsrooms do not run additional checks. As one foreign editor said:
AP says that when they put that material out from YouTube, they
have done the same verification process that they would do with
writing a wire story […so] we wouldn't do an additional verification
on that, because if we did that, we'd do it for every single story they
put out.
The strength of the agencies is their networks on the ground. Both the AP
and Reuters talked in detail about the role their regional bureaus play in
servicing content, and the importance of their local knowledge and language
expertise when verifying it. They are very aware of the need to talk to
the person who has supposedly shot the footage in order to strengthen the
verification process.

It is worth noting that there are different approaches to verification at the
AP and Reuters. The AP has standardized technical checks and processes
carried out in London after content has been discovered and filtered by the
regional bureau that found it. Reuters relies similarly on its bureaus but,
while the content is sent to London for cross-checking, there isn't a formal
procedure for verifying social content.
Some newsrooms, mainly public service broadcasting organizations in
Europe, do additional checks on content from the agencies. There was a
view among them that you cannot ``outsource verification.'' But even those
organizations that run additional verification checks on UGC from agencies
recognize that if that material has already been pre-vetted, the pre-vetting is
one consideration in their own verification process.
Verification and the Audience
Certainly none of the verification processes or checks undertaken by the
newsrooms were shared with the audience, either on television or online.
The only mention of verification was the phrase, ``These pictures cannot be
independently verified,'' which is heard very often when UGC is aired. We
heard a great deal of soul-searching about the idea.
On the one hand the AP has ``abolished the phrasing, ‘This cannot be independently
verified,' '' and a report published by the BBC Trust in the summer
of 2013 advised that it should not be used on screen or in script^{\href{#endnotes}{20}} (although
we saw some occurrences during our sample period). In other newsrooms,
however, it is a standard description, especially when referencing content
from Syria.
There is a shared awareness that because it is rarely possible to be 100 percent
certain about the veracity of a piece of UGC, this phrase acts as a type
of insurance policy in case it turns out that the content has been manipulated
or misattributed. Some editors and journalists actually saw it as a
mechanism of being honest with the audience. As one editor stated, ``Particularly where Syria is concerned, if we're not 100 percent sure that it is
what it purports to be then, yes, we will say [this cannot be independently
verified]. I have absolutely no issue, and neither does the channel, in being
honest with the audience, and I don't see that changing at all.''
But there was also concern that this phrase can frustrate the audience and
undermine trust, as it suggests that verification checks have been completed
inadequately, and fails to communicate the checks and internal newsroom
conversations that have taken place in deciding whether or not to use the
content. As one journalist said, ``I don't necessarily like that we have to say
it but you've got to. We trust our journalists, and we trust our contacts on
the ground enough to know that this is what it says it is, but we'll put the
caveat in.''
Another producer felt that with certain stories, if a video illustrated a pattern
that other sources confirmed, being unable to absolutely verify it in terms
of date or exact location didn't matter if it visualized something important:
I would say that, for example, with the barrel bombings [in Syria]
you can justify putting user-generated content onto a site and saying,
``We have not verified this is true, but we know that this is happening
all over the country and if we have videos of places being
barrel-bombed, we would say it is justifiable to push it out in that
way as long as you are informing people that they haven't been
100 percent verified.
There were different perceptions about the role of public verification, or
publishing content before it has been fully verified in the hope that the
crowd can help with the process. Andy Carvin made this form of crowdsourced
verification famous during the Arab Spring when he began using
Twitter as a mechanism for understanding what was happening on the
ground. The same idea of collaborative verification is now happening within
Storyful's Open Newsroom community on Google+.

One journalist raised this issue of publishing content before all verification
checks had been completed. He explained that in some situations, when
they believed it was in the public's interest to see certain images, a writeup
would be published and a link included to the unverified footage with a
disclaimer. It would then be updated with verification information when it
was completed. This was a rare position, however, as almost all other newsrooms
we spoke with were adamant they would not publish content unless
they believed it to be accurate.
Newsroom Pressure
Newsroom members regularly cited issues relating to the pressures they're
under to publish content before all verification checks are complete in our
interviews. One senior editor at the AP said, “I would always rather be last
to a story than first to be wrong to a story, and, you know, last to be right
with UGC is not a dishonorable place to be.''
While many shared this view, the pressure agencies face is slightly different
when it comes to newsrooms with audiences and competitors. As someone
who used to work on a newsgathering desk said:
There is still way too much pressure within news organizations to get
stuff up on air before it's properly verified, before the proper questions
have been asked, and there's just no excuse for that. And no
matter what anybody says, in any news organization that absolutely
exists and is an issue.
And this quote from someone who works on verifying content within a
large newsroom:

There's such pressure to get things on, especially if they're watching
the competition, and they're running with stuff, and so we have to be
really steadfast and put our foot down. Even though producers know
they should be verifying it, I can see them being overtaken by the
pressure to get it on air.
The pressure newsrooms feel they are under to ``tweet first, verify later''^{\href{#endnotes}{21}}
is a symptom of a news environment where a scoop today lasts 20 seconds
at most. You can no longer stay first for long, and as many commentators
have discussed, journalists are the ones who are obsessed with the notion;
audiences rarely notice.^{\href{#endnotes}{22}}
Conclusions
Overall, verification is the main area of concern when it comes to the topic of
user-generated content. However, despite this anxiety, very few newsrooms
are using systematic verification procedures and journalists feel they don't
have adequate expertise and would like more specific training on verification.
Managers are obsessed by the issue of verification, but have very little
knowledge of the specific technical checks that can support the editorial
information sourced from uploaders and local experts. Probably because
content sourced from social media did not exist when they last worked on a
news desk, there is a striking disconnect between managers and those who
work with user-generated content.
As one senior editor said, ``In terms of the verification processes, it's very
hard. We do our best, but every case is different. There's no system you can
set up that makes that work.''
In fact, as newsrooms are under more and more pressure, it is even more
important that there are systematic procedures in place that can provide clear
guidance to the output editors about which checks have been completed, and
the level of confirmation regarding specific facts about the footage.

The agencies currently play a critical role in verifying the content that
appears on people's television news screens. However, when the newsrooms
themselves have limited knowledge about the checks and procedures that
can be carried out around content sourced online, it makes it difficult for
them to ask questions of the agencies about those that have already been
carried out.
The pressure on people to publish quickly will only continue, but there
seems to be a growing recognition that newsrooms need to use verification
and context to differentiate themselves. Research by the Pew Research
Center in 2012 revealed that worldwide, YouTube is becoming a major platform
for viewing news. In 2011 and early 2012, the most searched term of
the month on YouTube was a news-related event five out of 15 months,
according to the company in early 2012.^{\href{#endnotes}{23}} With audiences increasingly seeking
out eyewitness footage on social media, news organizations have to distinguish
themselves. As one journalist admitted, ``People get [news-related
pictures and videos] on Twitter anyway, without the verification, so if you're
going to use it on air, what you're going to have to bring the audience is the
story behind [the pictures].''

\chapter{Permissions/Copyright}
``There's a Wild West attitude about getting stuff off the Internet'' was a
phrase that peppered our interviews. Most journalists, however, now know
that copyrights exist with uploaders even after they share it on a social network
and understand the need to seek specific permission to use someone's
content. Still, there's a difference between what people know and what
people do.^{\href{#endnotes}{24}}
Certainly the pressure of rolling news means that there are more situations
on 24-hour news channels where a senior editor will make the decision to
run with pictures without securing permission (knowing they will ``sort it
out'' retrospectively if necessary) than on daily bulletin programs. Broadcasters
working outside the pressures of rolling news explained that obtaining
permission from an uploader was mandatory before using content.
Online differentiates itself from television again because most websites
have the capability of directly embedding social content. There were mixed
responses about whether a news site has to seek permission before embedding
content. There is no legal precedent here and many people are aware
that this is a difficult space. The terms and conditions of the different social
networks mean their users have agreed that their content can be embedded
on different sites, but a number of journalists expressed disquiet about publishing
someone's Twitpic on their site via an embed code, since that person
will not even know it has happened.^{\href{#endnotes}{25}} In two separate interviews, people discussed
the ethics of embedding selfies women had taken for Cancer Research
UK's ``no-makeup selfies'' campaign.^{\href{#endnotes}{26}} The pictures were public because they

had been posted on Twitter, but publishing them on a news website seemed
to change the context considerably for the journalists to whom we spoke.
Some people suggested that they would like users to get an automated alert
via the social network if their content is embedded elsewhere.
Even online journalists, who admitted they sometimes didn't contact
uploaders before embedding their content, admitted this only happened
if the photo had little hard-news relevance. Most people who work with
UGC discussed the need to talk to the uploader on the phone, not only to
help with the verification process but also the newsgathering one, as people
often had other footage. As one website editor explained:
We would try and contact that person, not least to say, ``What else do
you see, what else was happening at the time, who else was there?''
We'd do it to get more journalism out of it, but generally speaking, if
it is a still, an Instagram still, we would just use the embed code. We
wouldn't feel obliged to contact them. We would only contact them
if we wanted to use it in a way that made it ours and we want to talk
to them about the story.
Overall, it is very rare that newsrooms pay for UGC. Many interviewees justified
this reality, stating that most uploaders don't care about payment. As
one senior manager explained, ``Occasionally people ask us for money. Nine
times out of 10 in the UGC space it's not about money, it's about attribution
and permission.''
However, there was an awareness that this is gradually changing as audiences
recognize the value of their content, and licensing companies spring
up, contacting uploaders and promising money either directly or via revenue-
share agreements. As one producer argued, ``I think what has changed
is that people are shooting stuff with ever greater quality with their phones
and now appreciate the value of what they've got.'' Some interviewees talked
about the need for the industry to look ahead at the long-term implications
of these trends.

We saw a consensus among interviewees that once UGC has been uploaded
to a social network it loses all value, as it can no longer be an exclusive for
a news organization. The example of the exclusive amateur footage secured
by ITN in the immediate aftermath of the murder of Lee Rigby in Woolwich,
London, in May of 2013 was referenced multiple times in interviews
with UK-based organizations. This UGC scoop won ITN the UK's Royal
Television Society award for Scoop of the Year for 2013. In announcing the
award the jury noted, ``When ITN broadcast the shocking pictures of the
murderer of Lee Rigby filmed by a bystander on a mobile phone, the team
were ahead of the pack.''^{\href{#endnotes}{27}} While no one interviewed could cite the exact figure
the British broadcaster paid for that exclusive, editors at different newsrooms
talked about how it had impacted their newsgathering practices in
terms of thinking about sending producers to a breaking news event ready
to spend money and present legal documents if necessary.
The process of securing permissions for use differed greatly. It ranged from
an online form written by company lawyers for an uploader to sign, to a simple
tweeted ``yes'' (as long as it was screen-grabbed for later proof). There
was a clear awareness of the tension between the need to secure rights in a
way that will stand up in court, and the realities of traumatized uploaders
sharing content on social networks during breaking news events, often in
situations where Internet connections are unreliable.
\section{Copyright Law}
Our research demonstrated that there are different forms of copyright law
in different jurisdictions, and there is also a great deal of confusion about
copyright law in general. One of our interviewees, for example, acknowledged
(guiltily) that he had taken content from YouTube in relation to the
Arab Spring because there is no copyright law in Iraq. Another interviewee
based in Asia described the way that ``laws that govern UGC and copyrights

in our region tend to be a bit murky.'' In Sweden, legal responsibility lies
with the overall editor of the program, who would be personally liable if a
mistake was made.
In Australia, our interviews demonstrated that ``there is a widespread belief
that if you are reporting anything as a news story—even if it's about a viral
video of a dog on a skateboard—then you have a legal right to use short
pieces of third-party content without payment [or permission], as long as
the sources are credited.'' In fact, as Alan Sunderland, head of editorial policy
for ABC Australia, explained, ``You would have the capacity to use it
under fair dealing only to the extent that it is absolutely necessary because
of its news value, and then only for a limited period, on the day it happens,
when it's absolutely relevant.''
British newsrooms can also use material under the ``fair dealing for the purposes
of reporting'' copyright exception. The BBC did so with ITN's footage
from Woolwich. But as one senior editor admitted, if you had great pictures
and people weren't getting back to you, ``You would just take it and stick it
on the air and fair-use it. You would, but you would also inform a lawyer.''
The one thing that troubled everyone was that the person from whom you
are attempting to secure permission might not be the copyright holder.
Many producers shared stories of uploaders saying, ``Yes, of course you
can use it. I didn't shoot it, but it's fine.'' Some producers also talked about
an even more confusing issue that frequently arose when they were trying
to secure permission to use photos from Facebook: Even though wedding
photographs or school photographs are uploaded by the people in the
photo, they don't own the copyrights. Those are owned by the professional
photographer who took the picture in the first palace. As the AP explained,
``Unless someone has taken a selfie and posted it to their social network,
when you ask them for permission, it's not actually their copyright to give.''

\section{Securing Permissions}
Copyright lawyers and people who work in rights departments understand
the need for exercising caution when seeking permission. We are all accustomed
to seeing journalists reach out to uploaders on Twitter during breaking
news events saying, ``Can we use your picture?'' Those that have been
trained will first ask, ``Did you take this picture?''
What is not clear for uploaders is how that picture is going to be used.
Technically, someone who uploads a three-minute video on YouTube can
complain that the video hasn't been used in its entirety. When the video is
taken down from YouTube, cut and edited into a wider package, the original
meaning of the video could be lost. In this case an uploader has the right to
complain, unless he or she granted permission for this to happen. Similarly,
an uploader might agree to his or her content being used by the program
that reaches out, not realizing that it could be used by any other news organization
under the same corporate umbrella, or even used by a completely
different news organization that happens to have a syndication deal in place
with the first. But these complexities are very difficult to spell out to uploaders,
many of whom have just been caught up in a breaking news event.
The specific ways that permission is sought were raised in every interview.
Some respondents were happy with a tweeted ``yes,'' while others require
signed documents. One interviewer described internal discussions about
this issue:
It was a vigorous discussion about whether a Twitter ``yes'' would be
enough. I was saying, ``No, it wouldn't be,'' and other people were saying,
``Well, yes, but a few years ago you would have said that an email
wouldn't be enough; you'd have wanted a fax, and before that you'd
have wanted it written with a quill pen and a stamp on it.'' So, I mean,
everything is evolving and things are changing.

Producers regularly using social media for newsgathering expressed the
difficulties of balancing the need for watertight legal protection with the
informal nature of social media. As one explained, ``If you are chatting with
someone via DM [Twitter direct messaging] and suddenly you're saying,
‘What is your email address?' and ‘Can I send you this form and can you
print it out, sign it, and scan it back?' That's just not going to happen.''
The AP and Reuters explained that they always need permission granted
before they will distribute content, but admitted that in very rare cases they
will use content when it has been impossible to contact the uploader. Fergus
Bell from the AP explained, ``If it's very, very newsworthy and we know that
it's just that they can't communicate at the moment and we don't suspect
that there would be a reason why they would prevent us from using it [we
will use it]. We will also follow up afterwards.''
As the messages posted on an Instagram account during Typhoon Haiyan
illustrate, journalists know they have to seek permission, but the pressures
of the job often conflict with the realities of people's lives when they are
caught up in a news event. When uploader Marcjan Maloon didn't reply
to the repeated requests of any journalists for four days, someone had to
remind them that it was unlikely he would even be able to reply—as ``there
was still no power in Tacloban City.''
Another significant problem is that uploaders themselves often don't know
their own rights, and don't understand enough about the news business—
particularly archive, distribution, and syndication elements. ``We've got to
make it so clear to them because [copyright] is not something that people
always understand,'' said one journalist.
Both AP and Reuters told us that they required permission via an email
exchange, and both have wording that has been signed off by their legal
departments. The agencies emphasized that because uploaders haven't necessarily
heard of them, and don't understand how content is regularly distributed,
they have to make sure they have explained the process fully.

An interesting side note: Some broadcasters might be using the agencies, not
just as an insurance policy in terms of verification, but also in terms of rights.
As someone from one of the agencies argued, ``I think part of the [decision to
run with pictures before getting clearance] is a risk calculation on the broadcasters'
part, because they know that we're working on getting the clearance,
so they're thinking, ‘The clearance will come. Let's just run it.' ''
\section{Embedding}
As the quotes in this section's introduction illustrate, permissions around
embedding social content are cause for some concern. While the terms and
conditions of Twitter, YouTube, and now Facebook explain that by accepting
them, the user grants permission for their content to be embedded by
other publishers, there is reason for this unease.
As one digital editor explained, ``I think it would be much better if we had a
sort of an industry standard, agreed guidelines about the legality [of embedding].
I don't think it has been tested in a court of law and I think people are
beginning to understand that you can ask two lawyers and they'll give you
two different answers.''
He went on to explain how every decision to use content is a calculated risk.
``So if someone got a shot of the bomb going off, I'm not going to use that
without permission because frankly it could be worth tens of thousands of
pounds. But if it's nice images of people sunning themselves by the seaside,
you know what, you might just take that risk a bit more.''
\section{Payment}
Paying for UGC isn't new. A senior editor recounted the following story
from an earlier career moment: ``I remember a ferry disaster, and I phoned
up the producer and said, ‘If you can find anybody who had a video camera

on board that ship, then buy the material.' I said, ‘You need a contract from
them because otherwise we don't own the copyright, but the contract can
be written on the back of a cigarette packet.'''
And while most people reiterated that very few uploaders want payment,
there was also a sense that any disputes could be cleared up after the event.
As one former producer admitted, ``Actually the rule has always been, ‘We've
always paid if it's a good enough picture, so if it comes to that we'll pay afterwards,
after the event'.''
Another producer from a broadcaster in another country said that they
didn't pay as a rule, explaining honestly, ``It would just become cumbersome
and unworkable.''
Payment also seems to be culturally dependent. While producers from the
United States, the United Kingdom, Australia, and Europe were adamant
that requests for payment were rare, a producer working in Africa said,
``When you do get content, it tends to be much more of a paid model where
they are journalists. Even if they're not in a steady job, they have an interest
in this type of thing. It would be rare for a guy to just witness something,
film it, upload it, and be happy for people to use it for free.'' There are sites
popping up in Africa encouraging people to photograph news events, and
then offering small amounts of money for these pictures.
Other journalists talked about the increasingly blurred line between citizen
journalists and freelancers, and the ethical implications of that. A managing
editor explained that more people are asking for money, because more
licensing companies are willing to pay. He explained, ``Now you'll suddenly
get, ‘Well so and so will pay me' or ‘Somebody will pay me x.' Well, fine,
you'll have to go to them because we won't pay. You're not trained, you're
not a journalist, you're not a freelancer and you're not someone we want to
take responsibility for.''

A foreign editor also talked about the increasing number of people who are
traveling to Syria and then upon return get in contact with news organizations.
``They get in touch with us and say, ‘I went in somewhere, you didn't
commission us, we're out again and here's what we've got.' Now, those are
still user-generated, but they are trying to make money out of that usergenerated
content.'' Editors discussed their serious concern about taking
this type of content because of the precedent it sets for freelancers who are
putting themselves in increasingly dangerous situations.
\section{Licensing}
Quite a few of our interviewees referenced the impact of the small licensing
companies springing up to manage UGC. They shared concerns that
most of these companies were not set up by journalists and were operating
without any sense of journalistic ethics. Many organizations were uncomfortable
about or even refused to use these companies for reasons of ethics
and uploader safety.
While the agency Storyful does license UGC, it has very strict guidelines
about not licensing videos that show gratuitous violence and death, or videos
that have been captured by people putting themselves in danger or
breaking the law. This is not the case for all licensing companies.
One producer noted another issue—that uploaders often don't understand
the term ``exclusive'':
Those agencies are a bit screwy. There was definitely a Woolwichrelated
photo where we got stung by an agency. It was one that was
a wide shot of the whole street scene, but you couldn't see any of the
detail. We had originally said to the guy, to the individual, ``Are you
okay for us to use it?'' He said, ``Yeah, whatever.'' And then a day or so
later, we got an invoice for £350. But we've got a screen grab of [our
Twitter conversation with him]. They [the agency] are still there say
ing it's fine to use it. So in that situation we're not going to pay [an
agency] for it. We took it down but we didn't pay for it. It's a bit of a
Wild West out there.
The world of licensing and UGC could run into an entirely separate section,
but it is worth recognizing here the emergence of different payment
models. Some agencies buy the copyright content outright, and resell it to
different broadcasters. Other agencies and publishers are using revenue
share models, whereby no money changes hands at the beginning of the
agreement, but when a piece of UGC generates views on a player—whether
on YouTube or a publisher's own player, which is surrounded by advertising
and has pre-roll ads before the video starts—the uploader, the licensing
agent, and the publisher come to a revenue-share arrangement. This is
clearly more appropriate for viral videos of talented babies or funny cats,
than it is for hard-news content.
\section{Distribution and Syndication}
There are real problems associated with the audience not understanding
news terminology or how their content can be distributed and used around
the world, especially when a person thinks he or she has just given permission
to a favorite news program. As one editor explained:
We work with lots of different partners around the world, so when
we get content in, we always say to people, ``Are you happy for it to
be used across all of our platforms?'' People are fine with this. But on
big stories, you get all of our partner news organizations saying, ``We
really want to use that UGC.'' Then what happens? The [terms and
conditions people get via email when they contact us] do explain that
our partners might use it but people often don't read that. So sometimes
we have to call people back and say, ``We've had a call from
Australia Broadcasting or Canada or European broadcasters, they
want to use your material. Are you happy with that?'' Often people

don't know what that means and just say yes. I wonder if that's an
area we need to think about. I wonder whether organizations need
to really think about their terms and conditions and revise them in
some way.
This isn't a new problem. George Holliday was the person who filmed the
Rodney King beating in 1991. He was encouraged by friends to pass on the
footage to Los Angeles TV station KTLA, which paid him \$500. In a story
in the LA Times from 2006, he explained how much he regretted that decision
when KTLA distributed the content to its networks, which played and
played the video. ``He didn't have kind words for the media. He may have
pioneered citizen journalism but he feels that he was swallowed up and spat
out by CNN and the like, which, he said gave him little credit and no compensation
of his contribution to history.''^{\href{#endnotes}{28}}
It could be argued that uploaders are becoming more astute and are certainly
in the scramble for permissions that occur when compelling content
is shared on social networks. Journalists are sometimes forced to justify
why they're not paying the uploaders. As an exchange in the immediate
aftermath of the Glasgow helicopter crash showed, journalists are having to
be more transparent about the news process to the people from whom they
are seeking permission. In the example below (FIGURE 5) a Reuters journalist
has to explain that the news agency's business model is subscriptionbased
rather than based on the sale of individual pieces of content.

FIGURE 5: Conversations Between a Reuters Journalist and Uploader Jan Hollands About Using Her Photo
Examples of journalists asking for permission via Twitter are seen during
every news event. Permission is almost always granted, but our research
found that credit was very rarely added to the content when it was used
onscreen or online. This can also be the case when UGC is distributed by
Reuters and AP. In the instance of the Glasgow helicopter crash, the agencies
advised of the need to credit the uploader (FIGURE 6) in the information

sheets distributed along with the content (the dopesheet). The broadcasters
in our sample did not uphold this request. The graphic below is taken from
the dopesheet associated with Christina O'Neill's photograph, which she
uploaded to Twitter.
FIGURE 6: The Dopesheet Circulated by Reuters with a Picture Sourced From Twitter
\section{Conclusions}
Marina Petrillo, editor-in-chief of Radio Popolare in Italy, has publicly
used an analogy comparing UGC to wallets that journalists pick up off the
ground. They take out the contents without even bothering to look for a
name inside, she infers. Those journalists who work with UGC every day
would argue that they certainly don't behave in this way. But one digital editor
did describe the mindset many journalists have, saying, ``I think people
tend to see it as, not ‘How can I nick it?' but ‘How can I use that on my [television
news bulletin or website]?' without actually thinking, ‘How do we use
it in a collaborative way?'''
The current method of journalists seeking permission via messages on
Twitter, YouTube, Facebook, and Instagram is laborious, legally dubious,
and can be stressful for people who are caught up in the middle of a breaking
news event. There are ideas circulating about embedding a breaking
news license into social networks so that if users opt in their content can
be used for free by news organizations for 24 hours.^{\href{#endnotes}{29}} Any subsequent use,
either in longer packages, documentaries, or within archives would require

specific permission from the uploader. In addition, if organizations wanted
to syndicate the content, they would need to establish a separate agreement
with the uploader.
There also appears to be a need for clearer guidance around different copyright
laws globally, as well as a better understanding of the legal implications
of embedding content without seeking explicit permission.
The news landscape is changing. There are growing numbers of licensing
agencies, more uploaders demanding payment for their content, and
increasingly blurred distinctions between citizen and accidental journalists.
As a result the industry needs more guidance, both in terms of legal advice
and ethical standards. Without these, many journalists feel that social newsgathering
is wild territory.

\chapter{Crediting}
There are two important factors when acknowledging user-generated content.
First, there is the issue of crediting, which refers to the practice of
naming the person who shot the footage, either onscreen, within the script,
or within a caption online. Second is the issue of labeling UGC, or signposting
to the audience that the pictures were not filmed by a person connected
to the news outlet.
Overall, there was an acknowledgement among interviewees that uploaders
probably should be credited, but the realities of the newsroom mean
that often they aren't. Our analysis showed that only 16 percent of UGC
included in the study had actively been given a credit by the newsrooms.
We had expected that there would be a few omissions in terms of crediting,
but were fairly confident most pieces of content would have some form a
credit. As one editor said, ``As a broadcaster we are founded on rights. We
are a rights holder ourselves. We need to respect people's rights to their
materials.'' So, the 16 percent was a surprise.
The explanations for why this number was so low included: the pressure of
output on a rolling news channel, concern about screen clutter, technical
workflows, and a belief that the television audience won't remember a credit
onscreen for more than five seconds so therefore it doesn't make sense.
There were differences between TV and online output, and interviews with
people who worked in the different areas demonstrated how the collaborative
nature of the Web has had an impact on the mindset of people who
work online. Those who have been television journalists all of their careers
were much more likely to question the need for crediting.

The legal requirement to credit was mentioned very infrequently, and only
by people who work in rights departments. These people are very aware
that newsrooms need to credit and are terrified that in the near future an
uploader will take a news organization to court for using content without
credit, thereby preventing future use of UGC.
Interestingly, most newsrooms are asking for permission before they use
content but are not transferring the permission that has been granted into
a credit. One copyright lawyer expressed real surprise at this practice,
explaining that in most copyright cases that end up in court today a person
will have been credited, but their permission won't have been sought.
The number of senior managers who hadn't given crediting proper thought,
and the absence of formal crediting guidelines, surprised us. Some interviewees
asked us why they should be crediting content. This attitude
shocked us, but was summed up very nicely by someone who works in a
rights department, who described the cultural tensions that exist between
legal teams and producers. ``Journalists just see [our focus on copyright] as,
‘You're stopping me. I am toiling at the coal face of truth here and you're
putting in all these things to make my life difficult.' So there is a bit of a cultural
thing there.''
Our interviews revealed many anecdotes about the difficulties of getting
credits onscreen due to different newsroom systems that often detach
crediting information from the image. They referenced the fact that many
default news templates don't have the crediting ``strap'' included (meaning
there isn't an automatic prompt to remind a producer to include a credit).
But as one producer concluded, all of these obstacles could be removed
and improved if journalists understood this as something that was nonnegotiable,
like sports rights.
Overall, the broadcasters credited only 16 percent of the UGC broadcast on
television during the three-week period we sampled. But this percentage is
an average and hides some real differences between broadcasters.

TABLE 6: Who Added Credits to the Content?^{\href{#endnotes}{30}}

As these numbers demonstrate, there are clear differences by broadcaster.
Fifty-three percent of the content broadcast by CNN International was
credited, compared with 15 percent by euronews, and 1 percent of Al
Jazeera English's content. It should be noted that CNN International and
Telesur have policies of crediting all pictures not filmed by their own cameras,
so they routinely credit Reuters, AP, and Getty Images. This ``habit'' of
crediting any external content demonstrates how newsroom culture has a
significant impact on practice. In newsrooms where any type of crediting is
rare, the checks required around UGC are not ingrained.
Similarly, UGC content that featured on France 24's Les Observateurs, a
program dedicated to the stories that emerge from UGC, credited uploaders
at the end of the program. In this segment the uploaders often feature
in the program themselves, and so their names are added to the end of the
show as credits alongside the producers. They are treated as partners in
making the program. This form of crediting was not included in our analysis
since the credit was detached from the content. As ever, statistics can
sometimes hide nuance.

As the table on page 80 illustrates, BBC World only credits 9 percent of
UGC on its television output, but 49 percent is credited online. Meanwhile,
euronews credits 15 percent on television and 13 percent online. It is
important to note that simply embedding a piece of content was not coded
as a news organization adding a credit, partly because many newsrooms
admitted they didn't seek permission if they embedded content so it seemed
inappropriate to consider this an active credit.
There were other types of credits that appeared onscreen, beyond those
added by the broadcaster. For example, some uploaders burn logos onto
videos or pictures themselves. Overall, 30 percent of the content we analyzed
from television had a watermarked logo burned on. This is perhaps
unsurprising, as this is the practice of most Syrian activists and 45 percent
of the UGC broadcast in that three-week period was about the Syria conflict.
38 pieces of UGC during this period had the watermark of a different
news organization that had burned its logo on when it bought a piece of
UGC. So, for example, during the broadcasts included in our sample BBC
World used UGC from the Woolwich attack to report on the court case that
was ongoing at the time. The credit on the picture used was The Sun newspaper,
which was the news organization that had purchased the UGC. The
Sun had burned in a large logo so it was guaranteed credit.
FIGURE 7: BBC World Used UGC Purchased by The Sun Newspaper

In early May of 2014 the Herald Sun, a newspaper owned by News Corp,
purchased images of a street brawl between Kerry Packer and his friend. So
worried were they that they might not be credited by other news organizations
the Herald Sun employees burned on their own watermarks in a way
that caused quite a lot of discussion online.^{\href{#endnotes}{31}} This was not a case involving
UGC, but shows how watermarking is seen as one of the only ways
to ensure credit. This has, for instance, long been the practice in Pakistan
where news channels regularly burn their logos onto all of their output for
fear of content theft.
FIGURE 8: Watermarked Pictures Published by the Herald Sun on Twitter
Notably, once a news organization buys a piece of UGC, news managers
were clear with us that the original uploader no longer had any right to be
named. As one senior editor stated, ``If you've bought [the pictures], then it's
our copyright, so we wouldn't see the need for a credit.''

\section{Reasons for Crediting}
Some of the people we interviewed were incredibly passionate about the
need for crediting, and expressed surprise that it was even an issue. As one
senior manager at an online news website in Asia exclaimed, ``If we didn't
credit, we would get crucified.''
Certainly the law in almost all jurisdictions in which we interviewed people
requires permission be sought, and if content is used, requires that it be
credited if the owner so wishes. However, this legal requirement was rarely
raised in any interviews. Many people did recognize, however, that most
uploaders simply want attribution. One manager explained, ``More people
want attribution than they want paying. So many big problems could go
away [if we credited properly].''
Others displayed confusion about why crediting uploaders was so different
from the way their own journalists would receive a byline. ``People have to
be credited, just like we credit our own sources, and our own journalists,
and our own programs.''
Most notable was recognition from a few people that the issue of crediting
is more than simply a legal or ethical one. It plays an important role in
signaling to the audience that journalists take UGC seriously. As one editor
explained:
I find it very helpful to be crediting UGC because we want to encourage
people to send us stuff. You have really got to make [uploaders]
feel like they're being paid back, otherwise they're not going to come
back to you. In the end, I think it's going to get very competitive and I
think you'll have a choice about where you're going to send your stuff.
So rather than thinking you're lucky your content is being shown on
[a global broadcaster], it will be, ``Who do I have a relationship with?''
I think organizations need to be quite careful now.

\section{Why are Credits Not Added?}
Interviewees offered a number of explanations for why most newsrooms
are not systematically crediting UGC: ignorance and confusion from journalists
and uploaders about rights; reliance on the news agencies; technological
barriers; the pressure of breaking news situations; concerns about
the screen clutter caused by crediting; and unease about crediting certain
organizations, especially within the Syrian context.
\begin{enumerate}
1. Ignorance
It was quite evident that many journalists and news managers don't understand
why crediting UGC is necessary and certainly don't consider it a
legal requirement. As one very senior manager declared, ``We don't credit
Reuters. We don't credit AP's pictures, so why would we credit user-generated
content when there is no requirement for us to do so? I'm not quite
sure what's the point of the credit. Is it to make somebody feel better about
the fact that their material is out there?''
Someone who does training at different newsrooms shared stories of many
journalists asking, ``If it's on YouTube and it's not a private video then we
can use it, right? If it has been on Facebook and it has been publicly posted,
then we can use it, yes?''
Another editor explained that part of the issue here is the lack of systemized
practice in terms of how UGC is used and credited, noting that every
conversation with an uploader is different:
It all depends, you might phone somebody up and say, ``Can we use
that video from YouTube?'' And they'll say, ``No problem.'' We'll ask if
they need a credit, and they'll say, ``No, just use it.'' The next person
you call, they'll say, ``I only want [the photo] on one bulletin. I don't

want to see it anywhere else, and I need credit.'' Or they may have
already burnt a credit in, and they'll say, ``I don't want you to obscure
my credit.'' It's very different depending on whom you speak to.
Even when newsrooms have crediting guidelines, they aren't what we
expected. One newsroom in Europe has clear guidelines that neither usernames
nor real names can be used. Instead, the policy is that UGC should
all be labeled: ``Source: Internet.''
Some newsrooms have formal guidelines on crediting, and others suggested
that while it isn't written down, there is an understanding of what journalists
and producers should be doing. The most common answer, however,
was: ``I don't honestly know whether we have guidelines.'' Considering the
number of pieces of content that weren't credited, there is certainly a disconnect
between what newsrooms think they should be doing, what managers
think is happening, and what is actually happening.
2. Role of the Agencies
One of the standout findings from our research is how reliant newsrooms
are on news agencies for discovering, verifying, and clearing the rights for
using UGC. However, for many newsrooms, when a piece of UGC enters
the newsroom via one of the traditional agencies or similar sources (e.g.,
AP, Reuters, AFP, Eurovision News Exchange) most journalists are unaware
they're working with UGC; instead, they think of it as agency ``vision.'' As
one journalist admitted, ``We always name any photo that doesn't belong to
the agencies.'' Others said that any vision from an agency wouldn't be credited
as a matter of course.
Given this blind reliance, the practices of the agencies themselves regarding
crediting are a crucial part of this equation. The AP always includes the
name of its contents' uploader, and any information it has about them. In
the case of Syria, the AP names the activist group and describes the type
of group they are, plus their affiliation. Storyful, and by extension Eurovi
sion which uses UGC sourced from Storyful, includes mandatory crediting
information. However, just because the need for onscreen crediting is
spelled out on the dopesheet,32 one senior AP manager admitted, ``We can't
guarantee that our clients will implement it.''
Reuters, by comparison, does not name sources. It lists the source of UGC
as ``social media website.'' Employees explained that if they source material
on the ground from a citizen journalist or accidental journalist via one
of Reuters' staff, they name them as a source because someone will have
spoken to them and made a connection. But when something is sourced
from the social Web, even though Reuters will have emailed the uploader to
secure permission, one of the senior editors explained:
[To us] this is still a social media video. We were not present in the
room with [the uploader], we have no prior relationship with that
person, so all our social media videos carry ``Material was obtained
from a social media website...'' It is a disclaimer. The source is ``Social
media website.'' … So for Reuters, our whole reputation depends on
our reporters and our camera people being on the spot or us having
a close relationship with the broadcaster who was on the spot. We
don't have that with the vast majority of UGC that we use, so we can't
say that... They might write it, ``Yes, I am the copyright holder. Yes, I
shot it; it was quarter to four in the morning,'' but they could be lying.
AFP follows this model and doesn't provide details about the uploader. This
is, therefore, one of the clearest reasons why so much of the UGC we examined
as part of this project was not credited. Most broadcasters rely very
heavily, and in some cases entirely, on the news agencies to supply UGC. If
the uploader information is not available via two of the largest agencies, this
explains why uploader information was not added onscreen.
Relying on agencies distances the newsrooms from uploaders themselves.
Our hypothesis is that by communicating with someone involved in a
breaking news situation, you are much more likely to think about giving

them credit. When you are removed from that process, and the pictures
simply look like any other vision in your gallery, the fact that it's UGC gets
lost and the related processes that should be followed get lost as well.
3. Newsroom Processes
There are many explanations why credits aren't added, but the reality of
outputting news demonstrates how difficult it currently is.
As one producer explained, ``I think it's a combination of workflow, technology,
the way different bits of kit talk to each other or don't talk to each other,
and the pressure of breaking news.''
Since the digitalization of newsrooms, Media Asset Management (MAM)
systems have become central to newsrooms' workflow. They are used
among other tasks to prepare rundowns, write scripts, edit video, create
on-air captions and, critically here, collate and distribute metadata for content
ingested by the organization as a whole. This includes UGC received in
all the forms discussed above. Metadata is the text data either written by an
organization to accompany its own-shot video, or distributed by an agency.
This includes the storyline; a precise shot list to describe the video; information
such as data and location shot; and restrictions, such as crediting
requirements, time of use embargoes, and so on.
While MAMs facilitate the sharing of content across newsrooms, allowing
any journalist to access, edit, and bring content to air, they have developed
without additional considerations for UGC ingest. They are designed so the
metadata, or dopesheet, travels with the content to journalists' desks. This
has been particularly important for organizations with a large amount of
output channels, such as the BBC. However, the MAM developers have not
yet found a failsafe way to ensure that metadata and restrictions—so crucial
for UGC—always accompany the content.

The BBC's MAM system, Jupiter, has a simple traffic light system to indicate
to journalists what content they can and cannot use. Green indicates BBC
content that is free to use across all output. Red is content that is not accessible
for a variety of reasons (embargoed, etc.). Amber is content received
from outside the BBC that can be used, but subject to checks. This includes
all content received from the agencies, as well as UGC.
One intake editor at the BBC highlighted a piece of UGC that showed students
attacking Prince Charles' car during protests in London in 2010. He
noted that any journalist who really wants to use content can do so—even
if it's marked red in the Jupiter system—admitting, ``You can do as much
metadata and marking [as you like], but if there's someone really determined
and they want to use that material, they can.''
Another journalist at the BBC also noted how the traffic light color disappears
when transferring video from Jupiter to the BBC's main editing system.
If a producer wishes to edit a large amount of content and didn't note
restrictions before transferring a collection of packages to the BBC's editing
system, it would be impossible to know after what content carries what
restriction. He noted that while experienced producers did not necessarily
fall foul of this, it was an easy mistake to make for junior producers, which
can lead to crediting information not being carried forward into the gallery.
Another producer noted, ``Even though we try and make it absolutely
important so it's flagged up in Jupiter, and it says MUST CREDIT in bold or
whatever, somehow that information doesn't travel with the video.''
Some of the people we interviewed spoke with real knowledge about how
technical changes need to be made if there is any hope of practices changing.
One example involved the visual templates producers choose for output:
The default [template], especially in a breaking news scenario, is to
use a full-frame still, which has no room for a credit, which then
tends to get used for hours. There is a template for a full-frame still,
which has a space to fill in a name and location, or name and date,

etc. So we could do with a rule that a producer using UGC should
use this template, in the first instance, and this would then only have
to be changed if someone didn't want to be named. The same rule
could apply to video, as there's an info tab where you can fill in the
``video courtesy of'' field which appears on the top right of the story.
Others admitted that when you're getting ready to edit a package, you pull
down all the vision you might use into the editing software. In the final cut,
you end up only using a small percentage of all the images you pulled down,
but if the credits aren't burned on at that stage, they get lost in that process
as a stretched producer isn't going to go back and search out the credits and
apply them retroactively.
Some organizations have been working on this problem to assist journalists
in attending to the information carried in metadata. RUV of Iceland, for
instance, has been exploring ways with its developers to transport metadata
and burn crediting directly into the video through its transcoders; there is
awareness at the BBC that more needs to be done technically with MAM
systems to help journalists avoid metadata errors.
4. Screen Clutter
The aesthetics of crediting was a recurring theme. The attitude of certain
editors came up often, including this quote: ``We don't credit, it's not
our style.''
One producer explained, ``A lot of senior editors don't like the way a name
under a photo looks. If the name is really long, they think it looks messy.
They'll say, ‘Can't you just make it look short like AFP/YouTube?' ''
Another senior manager asked whether crediting has any point when it
appears and disappears so quickly from a TV screen.

Let's say there's 20 pictures of a helicopter crash. Do you need to label
every single one of them? What would be the point of that? I don't
think [crediting] always has to happen. Even if you put ``picture by Joe
Bloggs,'' find a member of the audience 10 seconds later who knows
who took that picture.
He went on to discuss the differences between online and television viewers.
Online, people can stop and take time to look at credits, but on television
he offered the theory that there was little point in crediting as the
credit was on screen for such a short space of time.
Audience research is clearly needed here. Very little is known about how
audiences respond to the aesthetics of television broadcasts. Do onscreen
credits upset the viewer? Ultimately, however, this discussion about aesthetics
completely ignores the legal implications of this issue.
5. The Complexities of Syria
Certainly content from Syria caused the most discomfort in terms of crediting,
even from people who were the biggest advocates of the practice. Our
interviewees explained that the increasingly elaborate logos watermarked
onto content uploaded by Syrian activist groups did not signify anything
to the audience. However, they also felt uncomfortable spelling out the
names of these groups, because their motivation for sharing these videos
was clearly political.
As one journalist explained, ``I wouldn't be overly concerned about crediting
activist groups in Syria in that way because the issue is, in my mind,
not about credit. Basically, they are campaigning to show the world
what's happened.''
Another agreed: ``Mentioning the channel doesn't mean anything for the
audience. You can say it's Shaam News Network; it's not relevant anymore,
because they don't want the credits, they just want to air their video.''

\section{How to Credit}
Even when journalists want to credit, additional complications can hinder
their efforts. A small percentage of uploaders are very clear that they don't
want to be credited. Sometimes this is to ensure their personal safety, particularly
if they are sharing content from certain countries. Some people simply
didn't want to be named because they ``just feel they're doing a service.
They're not doing it for kudos.'' It's interesting to note that the AP has strict
anonymity policies, so just because someone prefers not to be named, it
doesn't necessarily mean that AP is willing to source content as anonymous.
Most newsrooms told us they preferred to use real names, rather than usernames;
some even noted some difficult cases in which people with long,
slightly silly usernames had given permission for their content to be used on
the proviso that their username be visible. This is sometimes reason enough
for news organizations not to use UGC. Other newsrooms wanted to use
usernames, admitting that doing so signaled to audiences that they were
using social networks and understand how they work.
\section{Crediting Online}
UGC used online was much more likely to be credited. Online has a culture
of crediting content, and the ability to embed content directly means there
is an implicit form of crediting in place even if no additional watermark or
caption is added.
As the editor of a UK news website explained, ``That's the great thing about
digital. It's much more collaborative because you can embed content, you
have photo expansion via Twitter embeds, that sort of thing. It actually
allows you to use a lot more UGC in a much more natural way.''
As will be discussed in the following section, almost all online journalists
admitted that they often didn't seek permission for embedding a piece of
content sourced from the social Web. If they wished to talk to the uploader

about what they had seen in the hopes of building out a story or verifying an
event, they would reach out and seek permission. But if it was simply a case
of embedding a picture from Twitter or a video from YouTube, the uploader
wasn't notified. In these situations, an embed is considered a form of credit,
but one that television newsrooms just couldn't physically do.
\section{Conclusions}
Journalists and their managers are not considering the legal implications of
not crediting UGC. Meanwhile, those working in legal and rights departments
are. They are desperate for staff to realize the seriousness of the issue.
As one person working in this area argued:
I think the issue is, the journalists basically think all this stuff is just
bullshit. It's just management bollocks when we try and say this stuff
to them. They think we're stifling their creativity, and they don't
understand that this could get them into hot water to such an extent
that we can't use [UGC] anymore and their creativity will be far
more stifled.
Certainly in our interviews it was very rare to hear people expressing concern
about not crediting uploaders. There was a sense that it is the right
thing to do, but an acceptance that in the heat of breaking news, crediting
is always a very low priority. There was also an acceptance that if uploaders
were unhappy that they weren't credited, or wanted payment, this could be
sorted out after the event. As one former journalist admitted:
You are in a massive sausage factory, under massive time pressures,
and there are fewer and fewer and fewer people to do the job… You
don't have time for anything, let alone worrying what somebody's
username is. You don't have time to think whether you should credit

their Twitter name or their real name, or who the hell they are anyway.
It's probably already in the system without the credits on it, and
you just use it.
The greatest frustration about the lack of crediting from broadcasters
comes from freelancers and pro-amateur photographers. During the London
riots in the summer of 2011, for example, many professional photographers
stepped outside their home, took pictures, and uploaded them to
social networks. When news organizations used these pictures without
credit, the uploaders took to blogs to express their disgust, explaining that
they didn't want payment (as it was a public service to document what had
happened during that period), but they were upset they hadn't received any
attribution. Three years later, and today it is just as likely the same thing
would happen.
Professional photographers understand their rights, whereas an accidental
journalist may not be aware that they are entitled to a credit. It is notable
that the only time copyright violation regarding a news organization distributing
content sourced from social media has been tested legally, the
ruling was in favor of the content creator, not the news organization publishing
the content. In November of 2013, a federal judge in the Southern
District of New York declared that Agence France-Presse and Getty
Images had infringed the copyright of professional photographer Daniel
Morel.^{\href{#endnotes}{33}} This came after the agency distributed eight photographs of the
Haiti earthquake in 2010 to its clients; Morel had originally distributed
them via Twitpic, a service that allows users to post images to Twitter easily.
A jury awarded Morel \$1.22 million in damages for the infringement.
Interestingly, this case was not cited in any of our interviews with news
managers or senior editors.
Indeed, professional freelance photographers and citizen journalists are
starting to lead the campaign for better crediting. John McHugh and Tim
Pool have separately built technology that automatically watermarks photographs
with a credit (Marksta and Taggly, respectively) to ensure photo
graphs will contain a credit even if newsrooms don't add one. Perhaps it was
their intimate understanding of the news business that persuaded them to
lend uploaders support, rather than try to convince newsrooms to make a
culture change.
It seems that only when senior management sends signals that crediting
UGC is as important as the rights restrictions placed on sports events that
behavior will change. The interviews made clear that there are technical
issues that impact why pictures aren't credited. These will only be amended
when the importance of crediting is highlighted.
As one producer admitted, ``At the moment it's all a bit ad hoc, so if the
bosses want consistency, people really need to know that it's something
they must do, not something that's nice to do if you remember.''

\chapter{Labeling}
Crediting involves providing details about the person who captured the
piece of UGC. Labeling, on the other hand, simply involves acknowledging
that the content is user-generated. On this topic there are two different
schools of thought. Some newsrooms want to be very clear that someone
unconnected to the newsroom filmed the footage. They surmise that by
labeling pictures as ``Amateur Footage'' or ``Activist Footage'' there is a safety
net in place if something goes wrong. They also believe it is important to
be transparent with the audience about where the footage has come from
to ensure that if the uploader has a particular agenda, it is clear to the audience.
Others believe it is obvious when the pictures are created by someone
unconnected to the newsroom, and separate labels are of little worth and
certainly don't represent an insurance policy that would stand up in court.
Overall, our study showed that 72 percent of content did not have any form
of description, be it a label (written or spoken) such as ``amateur footage,''
``activist video,'' or even the unfortunate term, ``source: YouTube.com.'' There
was certainly no uniformity about the way UGC is described to audiences.
And there was definitely no agreement about what constitutes best practice.
There was a general reluctance for onscreen labels, with more in favor of
including the fact that the pictures were sourced from the Web in the spoken
script. There was also an awareness that transparency with the audience
is crucial, but uncertainty about how best to provide it.^{\href{#endnotes}{34}}

As part of the analysis, we examined whether each piece of UGC was
described as content produced by someone unrelated to the newsroom (i.e.,
not a professional journalist). As discussed above, this took a number of
forms (e.g., onscreen captions of ``amateur footage,'' ``activist video,'' or ``You-
Tube.com''). Sometimes UGC content was obvious when it was embedded
online, rendering a picture from Twitter or a video from YouTube evident.
This idea of referencing a social platform received mixed reviews. Some
credit ``YouTube.com: Jane Smith,'' which automatically shows that it's UGC
as well as offers a credit. But others were very wary of referencing social
platforms, both due to concerns about overtly promoting commercial companies
and navigating content that often sits on multiple platforms.
One news organization detailed a clear policy on crediting both the platform
and the username, describing how the organization's head of legal had
been very clear with them:
If you're going to use something that was filmed when you weren't
there and no one you know was there, you can't 100 percent ever say
it is a fact. You just have to be belt and braces and say, ``This is from
this source.'' It means a) you're giving the person the credit where
credit is due, and b) if it comes back to you, we did say this was not
our cameraman.
Our numbers suggest that while this view might be the one shared by heads
of legal and rights departments, many have not been successful in translating
this into newsroom practice. As TABLE 7 illustrates, on television the
majority of UGC was not described or labeled specifically as content that
had been created by someone unrelated to the newsroom. It is worth noting
that it is much clearer on websites that content has been sourced from
the social Web, and is therefore UGC. This is due to the structural character
that exists online. This makes it possible to embed content directly

from social networks like Twitter, YouTube, Instagram, or Vine. When the
descriptions are compared by channel, it was clear that some channels were
more likely to describe content as UGC than others.
TABLE 7: Percentage of UGC Not Labeled as UGC

The fact that content had been sourced from the social Web was much
clearer on the programs dedicated to social media that feature on all rolling
news channels now. On Al Jazeera English, it was The Stream, on BBC
World, it was BBC Trending, on France 24 there are two: Les Observateurs
and Sur Le Net. These programs would play a YouTube video showing the
full YouTube page. We counted this as a decryption of the provenance of
a piece of content. And sometimes there wouldn't be a specific caption
describing the content, but the reporter or presenter would use language
such as, ``These pictures have emerged online.'' This was also counted as a
description of the footage as UGC.
TABLE 8 illustrates how different broadcasters performed in terms of labeling
UGC. The differences were quite stark. For example, 51 percent of CNN
International content on television was not described explicitly as UGC.
Telesur didn't label any content as UGC and NHK World failed to label 97
percent of the UGC it used. Online, some organizations were more likely
to label content as UGC in some way. On the CNN International website,
only 13 percent of the 450 pieces of UGC were not clearly labeled as UGC.
In contrast, 45 percent of UGC content on BBC World online had nothing
to describe the content as UGC in any form.

TABLE 8: Who Added Descriptive Labels to UGC?

FIGURE 9: Screen grab From a Broadcast by euronews During our Sample Period


In comparison, Al Jazeera English used the caption, ``YouTube.com/Activist
Video,'' which interviewees explained had been integrated into newsroom
guidelines a few months prior.

FIGURE 10: Screen grab From a Broadcast by Al Jazeera English During our Sample Period
\section{Should News Organizations Label?}
While most interviewees agreed with the principle of crediting (even if it
wasn't being done systematically), the issue of labeling was less clear-cut. As
one social media manager noted, ``I think crediting is a no brainer, but [with
labeling] I think you've got to make an editorial judgment about whether it's
necessary.'' Another senior editor at a different news organization argued:
I think honestly whether or not you put amateur, most people can tell
the difference. If you're using amateur footage from Syria, come on,
it's been several years. Do you need to put amateur footage on every
single image coming from Syria? Our audience is not dumb; they
are not stupid. They know it's not a journalist who shot the image,
so I think it's kind of redundant that we have to put it as a rule. Say
it's rule number 25 [that you always have to label]. I don't think that
works. I think it's important to do it when it's not clear.
There is obviously uncertainly about whether labeling is necessary, and if
so, how to signpost to a television audience that a piece of content has been
captured by someone unrelated to the newsroom. But there was an overall

sense that people who capture and upload content often have an agenda,
and that by not labeling, you're not being transparent with your audience.
I think it's all about transparency and I'm sure viewers and readers
don't like things that aren't transparent. When it's from UNHCR, I
think they need to know it's from UNHCR. Or when it's from someone
who happened to be on holiday in Italy who has this amazing
picture of a flood, they need to know that it's from that person
because I think it changes the perspective of what we're looking at.
So therefore I think it's very, very important that we know the source.
For some it was a quality issue: ``The argument I use to everyone is, from an
editorial point of view, do you not want to point out to people that this is
not our material?'' For others, it was an insurance policy:
In many of these cases, it might be the best defense you have. As
in, we put it through the usual checks, we checked it as much as we
possibly can, everything we normally do, and it still turned out to be
wrong. But at least we said, ``This is what this picture says it is.'' I think
this helps to some degree, but it doesn't totally excuse you.
\section{The Complexity of Syria}
Almost all interviewees talked about their discomfort with having to rely
so heavily on UGC from Syria, and struggled with identifying the best way
to inform the audience about the activist groups that were uploading this
content to YouTube.
Al Jazeera made a conscious editorial decision to label videos from Syria as
``YouTube.com/Activist Video.'' As one of its presenters argued, ``We used to
just say, ‘YouTube.com.' Now it's ‘YouTube.com/Activist Video.' That was a

conscious decision made. This is activist video. Take it. We will explain it as
best we can, but remember where it came from and I think that was a very
good thing to do.''
The AP includes descriptions about the different organizations in
its dopesheets:
At the time that we arranged these relationships [with the different
activist groups] we asked how they wanted to be referenced.
For example, Ugarit News has the big logo that looked like the Sky
News logo when it started, and so it was important to say that it was
released by a group that calls itself Ugarit News and then to tell clients
that this is not a recognized newsgathering organization. Similarly,
I remember Shaam News Network had to [run] through our
standards center as well. We ended up describing it as a loosely organized
anti-Assad group based in and out of Syrian territory. They
claim they do not have any connection to Syrian opposition parties
or other states.
\section{How to Label UGC}
There is no consistency across the industry about how to label UGC. The
phrase amateur video appears to be losing favor. ``Now we don't use [the
label] ‘amateur video' because, frankly, a lot of it isn't amateur and I just
don't think that it encourages people. If people have captured the most
compelling shots of a story and it's running everywhere, I don't think we
should call them amateur.''
Most people recognize that simply writing ``Source: YouTube'' or ``Source:
Twitter'' is inappropriate as well, arguing that these are not the sources, they
are simply platforms. However, there is a great deal of disagreement about
whether a news organization should use the name of the platform as part of
the credit. The AP does not reference the platform, arguing it is simply the

delivery mechanism and will only reference the platform if it believes it is
relevant; for example, if the content has been live-streamed via a service like
Bambuser or UStream. In this case, the AP would argue that the technology
of the platform is part of how the UGC was created.
CNN International dislikes referencing the platform, reasoning that the
same content can appear on multiple platforms, so it cannot reference
just one. BBC World does not like to reference the platform as its editorial
guidelines prevent undue prominence being given to a commercial entity.
Broadcasters in France are not allowed to mention Facebook, Twitter, or
YouTube on air for the same reason. The fact that some interviewees talked
about uploaders wanting their specific YouTube usernames used on screen
hoping that it would drive clicks to their video and therefore increase the
money they would receive from YouTube, shows how newsrooms can be a
conduit for traffic to commercial platforms.
However, many broadcasters with whom we spoke said they would mention
the platform and a name, for example: Twitter/Jane Smith or Twitter.
com/jsmith7564. By doing this, the broadcasters are providing a credit, but
they are also signposting the fact that the content was not filmed by anyone
related to the news organization.
\section{Conclusions}
As the discussion about the definition of UGC at the beginning of this
report demonstrates, the term includes many different types of content. It
might be footage filmed by a holiday-maker caught up in a breaking news
event, a video shot by a Syrian activist group, a photo of refugees crossing
the Jordanian border captured by a UNHCR staff member on an iPhone, or
an anonymous viral video that has actually been created by a PR company
for the launch of a new product. Simply labeling content ``Amateur Footage''
or ``Source: Internet'' does not capture these differences.

As one manager commented, ``I don't believe you can be too transparent,
and I don't believe there can be too much disclosure.'' However, the question
remains how to share this information with the audience. The Web makes
all of this much easier, but with television it is more of a challenge, particularly
when different regulations govern public service versus commercial
broadcasters, and different countries have different legislation relating to
social networks. Until someone is able to find a tag that adequately describes
footage that has been captured by someone outside of the newsroom, these
issues will remain. Fergus Bell from the AP has suggested the term ``External
Content,'' or ``ExCon'' for short. Maybe that will catch on.
Most important is the need to describe who has uploaded the content, with
a description of why their motivations should be taken into account by the
audience. A video uploaded by Greenpeace or Shaam News Network should
be treated very differently than one sourced from a passerby who happened
to capture a breaking news event. However, the best way of accomplishing
this is yet to be agreed upon across the industry.

\chapter{Responsibilities}
There are three main responsibilities newsrooms consider in relation to the
use of user-generated content. Their responsibility to: uploaders, the audience,
and their staff. We discuss them below separately, as they each involve
different issues and concerns.
\section{1. Responsibility Toward Uploaders}
Health and Safety
Overall, among the journalists with whom we spoke, there was a real awareness
of the health and safety implications related to uploaders putting themselves
in harm's way. Many journalists talked about the need to phrase calls
to action carefully, so it didn't appear they were encouraging people to put
themselves in danger. They also talked about the need to be careful that the
request didn't look like their news organization was commissioning people
to film for it.
Fran Unsworth, deputy director of news at the BBC, shared the organization's
experience during the Buncefield blaze of 2005, when a huge fire broke
out at a fuel depot just outside London. Local teenagers got very close to the
fire to film and, having been told by BBC producers on the ground that their
pictures were ``too wobbly,'' leapt up and announced they would go and get
better ones. They were told not to do so, as they were putting themselves

in danger. This forced the BBC to rethink its processes around UGC, and it
rolled out a specific training course regarding working with user-generated
content and uploaders.
Caroline Bannock, who works on the GuardianWitness project, explained
how her organization has changed the phrasing of its calls to action, discarding
``send us your pictures'' in favor of ``share your pictures with us.''
She continued:
I'm actually quite careful in a protest, so if someone is sending in
photographs of someone doing something that they could be picked
up for by security services, I won't publish that. These people aren't
journalists; they're sending us snapshots and they're sending us stories.
They don't have that sort of journalistic sensibility. So we're
particularly careful.
Similarly, the AP talked about the specific dangers of geo-location technology.
Fergus Bell explained, ``We didn't use some Bambuser live streams
because I could work out on Google Maps where they were and we were not
going to put that out live. Because if I know where they are, if I can pinpoint
the roof that they're on, other people can too.'' Despite some examples of
good practice, there was also the sense that good pictures were good pictures,
and while news entities wouldn't actively encourage people to put
themselves in danger, if good pictures came in, most would use them.
And while a large section of this report talks about the importance of seeking
permission, there was certainly an understanding by most that, when
working with uploaders from certain countries, not seeking permission is
the right thing to do. One BBC journalist working on a photo gallery from
Iran told us, ``As someone from the BBC it really raises a person's profile if
they've posted the image, by me saying, ‘Hello, can I use it? I'm from the
BBC.' So in that instance the Persian service advised that it's better to just
use it.''

Some people have given a great deal of thought to the blurring line between
freelancers and citizen journalists. Increasingly news organizations are stating
they won't use freelancers in Syria because of a concern about inexperienced
journalists taking serious risks without being insured. A few believed
the same issues were emerging with amateurs as well.
One foreign editor talked passionately about the potential impact of these
new licensing agencies on citizen journalists:
There are agencies popping up [and] encouraging people to film and
send them material. [Those people] then edit it and send it out and
say, ``Here are some images of an incident that took place yesterday
in Tahrir Square,'' or whatever. Now, I'm nervous and wary of them
because you're saying to people, if you go and film in dangerous
places, we'll give you some money, and if we sell it on, we'll give you
some of the benefit. Now that seems to me no different from commissioning
someone to go to a nasty place when you're not prepared
to go yourself. So you have a duty of care over the people who are
shooting that, so I don't use any of those.
There was certainly a wariness around providing technology to people
on the ground, particularly in areas where it would be impossible to gain
access. The case of the 18-year-old boy who died in Syria, who had been
taking photographs on a camera supplied by Reuters, was mentioned as a
warning to all news organizations about the responsibilities of the industry
to protect and support citizen and accidental journalists.^{\href{#endnotes}{35}}
\section{Privacy}
Facebook was mentioned as the platform that causes the most editorial
discussions around whether to use content, particularly after death. Some
journalists talked about the agencies that have emerged, which scour Facebook
for photos after someone has died, that they copy and then sell onto

newsrooms desperate for images. Reuters has a blanket policy of not using
pictures from Facebook. As an employee explained, ``Our pictures colleagues
don't touch Facebook at all. It's not about copyright; it's because it's
something personal.''
This explanation from a journalist at another newsroom shows that some
organizations will use photos sourced from Facebook, but with caution:
When someone has died we don't have the copyright so we're taking
an educated risk, and we would usually just use one or two images
that portray that person in, if not a good light, a light to which you
would expect the family to be happy. So no pictures of them from
their Facebook page throwing up on a drunken night out, no. Just a
picture of them looking nice, yes. We wouldn't start building galleries
based on their Facebook page and things like that at all; so one or
two images.
There were some examples of guidelines which specifically make mention
of the need to consider people's intent when they published content on a
social network. A couple of interviewees made the point that as more journalists
use social media themselves, they can understand these ethical challenges
with more nuance. A few years ago a journalist might have fought to
use content that had been posted to Facebook, arguing that it is public. Now
people are more aware of the complexities of privacy settings and might
have more sympathy with the uploader.
And finally, journalists discussed the ways that uploaders are sometimes not
ready for what happens after they agree for their content to be used. As one
editor at Storyful explained, ``One thing we're taking seriously is advising
people that their content is going to be seen all over the world. So we ask if
there is anything about it, like if they're swearing or the way in which they
react to what they're seeing, that they're not happy about. We advise them
to think about these things before we distribute their content.''

This type of comment wasn't necessarily common, but again, those journalists
who work with uploaders on a regular basis have seen many examples
of the impact taking a picture or video can have when it is picked up by the
news industry. Journalists who speak to uploaders regularly talked about
building a relationship with them and subsequently having a heightened
sense of responsibility toward them.
\section{Ethics of Public Newsgathering}
During breaking news events, blog posts often emerge about insensitive
journalists using social media to chase people caught up in the action. In our
interviews, a number of journalists talked about the difficulties of undertaking
public newsgathering via social media when the people who have shot
the footage are often very traumatized by what they've seen. Some journalists
who work regularly with UGC talked about the specific skills required
in these situations. As one former journalist argued:
My personal view when you're talking to people who have been in a
traumatic situation and you're asking for their photos, the first question
must be, ``Are you okay and are you safe? And don't put yourself
in danger.'' I've seen too many examples of journalists who do just disregard
that. They seem to forget that these people are traumatized,
or potentially traumatized, or in an extremely dangerous situation.
Some people shared experiences of talking with uploaders directly after an
event. One journalist who was working the night of the Glasgow helicopter
crash described some of the eyewitnesses he spoke to as ``befuddled,'' but as
he explained, ``You'd expect that as they'd just seen a helicopter crash into
the roof of the pub they were in.'' Another journalist described talking to a
man who had witnessed a train crash. She said, ``I realized then that when
people see awful things they don't necessarily act in a way that is rational.''

There is certainly a distinction between those seeking content and users
who find themselves unexpectedly in a breaking news event. A journalist
who works daily with UGC talked about the difficulties of building a relationship
with someone via a 140-character tweet. ``We don't actually have
a policy specifically around [social newsgathering] except to be as polite
as possible. You see people just hammering [uploaders]. ‘Call me. Call me.
Here's my number.' What we try to do is be softer about it, but it's still the
same. You're still reaching out to them.''
We therefore asked journalists whether they were given any guidance or
training about public newsgathering, and the specifics of how to reach out
to seek permission from someone in shock. The BBC's UGC Hub has regular
support sessions for staff and even training on how to talk to someone
who is traumatized; staff welcomed the chance to develop this skill.
For most people we interviewed, there was a sense that the ethical issues
that surround the use of UGC are no different than other types of ethical
decisions. Many interviewees argued that journalists simply needed to use
appropriate judgment when it came to using UGC. As a senior producer at
the AP explained, ``Through training we've tried to get people to adapt their
instincts to social, so if you were going to knock on the door of a loved one,
how would you act? If you're going to tweet a loved one, how would you act?
It should be the same way.''
Too often conversations around user-generated content are all about the
content, and the user is not considered. This is certainly the case because
so few journalists actually work directly with uploaders. They work with the
content once it has been discovered, verified, and cleared by someone else
in their newsroom, or by an agency. Throughout our interviews, those who
worked regularly with uploaders had very different ideas and views about
appropriate practice. For example, someone from an agency argued, ``I think
a big ethical issue is—what are the rights of the owner of the content, not

just the rights of the content itself? What are their rights to privacy? What
are their rights to smart advice upfront? What are their rights to amend or
adjust or even edit the video before it's distributed?''
It was rare that these types of moral rights were raised in our interviews.
Issues around health and safety have permeated through newsrooms, but
these other aspects have not yet been considered in great detail.
\section{2. Responsibility to the Audience}
Distressing Images
Secondly, we discussed at length the responsibility news organizations have
to protect the audience from violence and trauma. There was a sense that
decisions about showing graphic footage are the same now as they've always
been, but some people we interviewed even believed that social media is
pushing boundaries of what it is acceptable to show on television screens.
Editors from Reuters discussed how there are ``different clusters of broadcasters
who have different viewpoints on what they can and cannot run or
what their audience wants to see,'' with some countries being very conservative
and others happily broadcasting pictures that would be shocking to a
British audience accustomed to regulation by Ofcom.^{\href{#endnotes}{36}}
In many countries television news is governed by regulatory bodies, which
have clear guidelines about the types of pictures that can be shown. However,
the same regulations don't exist for online news sites, although many
newsrooms have very detailed conversations about what is or is not appropriate
to show on the Web. There seems to be evidence that some newsrooms
are pushing their own boundaries, acknowledging that sites like
YouTube allow audiences to access images that previously would only have
been seen by journalists working on picture desks.

The most commonly cited example was the chemical weapons attack in
Syria in August of 2013. Many newsrooms believed they had a right to show
some of that footage, but acknowledged the need to make it difficult to find.
So Channel 4 News, for example, has designed clickable online barriers
to reduce the likelihood of children stumbling across graphic images. As
its Web editor explained, ``That was a sea change [for us] because previously
the lawyers at ITN and senior editorial people would just say, ‘No way,
you can't do that, someone will complain.' Things have shifted because of
social media.''
Deborah Rayner from CNN agreed:
Pictures will be going viral and everybody will be seeing them. The
audience expects to look to CNN and see them because they look to
you for context when there's this type of footage. That's increasingly
what people look to the main broadcasters and publishers for; it's the
context. So sometimes [a picture] will be held up for quite some time
while senior editorial staff argues about whether it can be used, then
how it's used. So the availability of unexpurgated material on social
media does cause us ethical problems.
There is a need for more audience research on this topic of broadcasting
graphic images in a context when they are often already available online.
When British news bulletins broadcast the footage filmed by a passerby
after the murder of the soldier Lee Rigby in Woolwich, the BBC, ITV, and
the UK communications regulatory body, Ofcom, received roughly 800
complaints.^{\href{#endnotes}{37}} However, as one of our interviewees discussed, it is unlikely
that a traditional news camera would have been able to capture this type
of footage:

So, how would the audience feel about not having seen that footage?
For all the complaints we had, what is their view about [the news
industry getting that footage]? Is that something they do want in
their living rooms and we should be providing? Or is it something
they could live without?
Need for a Standardized Set of Guidelines?
The jury is certainly out about whether a common set of principles or
guidelines would work in terms of user-generated content. One senior
manager argued:
To be absolutely frank, what I don't think is useful is a kind of common
industry standard, which I know Storyful has been talking
about, because we all have different standards. I mean, for a brief part
of my career I was consulting and it didn't take me long to realize that
news standards and ethics are completely different country by country.
I just don't think trying to put one set of values or one set of filters
on this stuff is appropriate because story by story it will change.
However, other people argued very forcibly that a standardized approach
would be useful. As one journalist said:
I think we need to build an ethics code. It's like the Wild West out
there and I don't think we've addressed it properly. So everyone's
doing their little bit in different ways and I think actually what would
be really good would be all organizations coming together and really
thinking of a sensible way of treating UGC… I think some organizations
use UGC in a very different way from others. Some people
would take a photo from a Facebook page without asking for it and
other organizations wouldn't. So therefore it's very good for the public
to know what ought to be the standard as opposed to, ``Oh, they
did that and I don't know if that's right or not.''

\section{3. Responsibility to Staff}
Vicarious Trauma
In terms of protecting staff, there were real differences in the comments
made between people who work with this content every day and those who
don't. Those who do social newsgathering regularly were very vocal about
the specific trauma that can come as a result of working with eyewitness
content day in and day out. As one journalist admitted, ``I now purposefully
avoid the most traumatic, dramatic content because I have been so affected
by it. I know it's my job, [but] it's a difficult line to cross. I feel like it's my
responsibility… But at the same time I have to protect myself from it.'' The
interviewee went on to describe the impact UGC can have on a journalist:
I was really upset all the time and I just felt helpless basically. I didn't
really sleep that well and I was anxious at work. Each morning, getting
up and thinking, ``My God, I know what the day has ahead for
me—blood, children, death, whatever it is.'' I just didn't want to do it.
It was a kind of constant anxiety.
Some described the way wearing headphones heightened their reactions to
disturbing content, and the tension caused by opening video files on You-
Tube with no sense of what they would find. One experienced journalist was
quite adamant that watching content via social networks was different than
watching raw footage from the field:
What they haven't been looking at is video with headphones on of
mothers crying as they're burying small children. And if you're listening
to that all day, I've found it's not necessarily the graphic stuff
that gets you, it's being in the world where you're hearing everyone's
distress. And hearing that for an extended period of time is the bit
that can get to you. So I disagree with the point that we've all been
looking at this for a long time. No one has been immersed in quite
the same way as journalists working with social media today.

For many, it was the scale of the violent videos that have been coming out
of Syria for the past three years that has caused people problems, whether
that was difficulty sleeping, recurring images popping into their minds,
lack of concentration, or more serious emotional responses and depression.
Others agreed with the journalist quoted above, saying they could
cope with graphic images but struggled to ``hear'' constant audio of people
in physical and emotional pain. Others said that Syrian content was not a
problem, but had found that a news event like the Aurora movie theater
shootings in Colorado in 2012 triggered a response because it mirrored
their own life experiences.
Those who don't work with UGC often struggled to see how viewing it is
different than viewing any type of rushes from the field. ``I've been watching
graphic footage since 1988. I don't think there's any difference,'' said one
senior editor. It must be noted that even people who felt that viewing UGC
online was not any different from previous journalism jobs still work in
newsrooms where there is counseling support for anyone who needs it. ``As
a senior manager it's something you have got to take seriously. You've got a
duty of care to your staff.'' There were a range of responses on the issue, with
some broadcasters having active policies about rotating staff on UGC desks,
and actively reminding staff about counseling. Others didn't have specific
policies but agreed that if anyone demonstrated signs of vicarious trauma,
they would be recommended to a counselor.
The subject of vicarious trauma in relation to viewing UGC is starting to
become one of psychological study. In a forthcoming article, soon to be
published in the Journal of the Royal Society of Medicine, Professor Anthony
Feinstein and his two colleagues examined the potential impact of viewing
UGC in a newsroom. One hundred and sixteen English-speaking journalists,
who work frequently with UGC, provided the researchers with selfreported
measures that they were able to share anonymously. The article's
authors discovered that ``frequency of exposure to UGC independently and
consistently predicted multiple indices of psychopathology, be they related
to anxiety, depression, PTSD, or alcohol consumption.'' The research also

demonstrated that frequency, rather than duration of exposure to images of
graphic violence is more emotionally distressing to journalists working with
user-generated content.^{\href{#endnotes}{38}}
The BBC, because of its number of journalists working with UGC every day
on the Hub, takes vicarious trauma very seriously. As its manager, Chris
Hamilton, explained:
It's not that massive elaborate procedures need to be put in place.
We just need to keep reminding the team when there have been big
traumatic stories, that it's okay to feel affected. It's okay to say, ``Can I
work on something else today?'' It's about getting the people running
the desk day to day to try to bear in mind, has someone been working
on that story for multiple days? That's one of the risk stories. It's okay
to be working on a traumatic story for one day, but on the second day
that's where the risk factor starts to rise. It's just getting people to be
aware of the little steps they need to take.
Other newsrooms shared best practices with us. At the AP, for example,
staff is told that they can find support materials on the Intranet, as well as
book an appointment with a counselor; they don't have to feel they will be
judged if they ask for help. At ARD in Germany, there are regular lunchtime
sessions where specific techniques for minimizing harm are shared, so all
staff benefit, not just those who have asked for support. Other newsrooms
have a policy of rotating people off certain stories, even if they ask to continue
working on them.
As well as these discussions about the impact of viewing distressing images,
others talked about the impact of regularly talking with people who are traumatized.
Those working on UGC desks are, by the very nature of their jobs,
often dealing with people who have just witnessed something newsworthy—
violent protest, a natural disaster, an explosion, or terrorist incident.

Talking to people who have just seen such events causes its own trauma. To
counter this, some newsrooms have created support networks for journalists
working in this space to share experiences and best practices.
Training
There is very little specific training regarding UGC. There might be social
media training that teaches people to find content or, more commonly,
how to use social media to promote it, but minimal specific training is
dedicated to verification or copyright law. There are exceptions. The AP
hosts a one-hour training course on verification processes. (This is mandatory
for new starters and many other staff members have been through the
training. The systems are also included in the AP Stylebook.) ABC Australia
has something similar, which is mandatory for anyone working in roles
where they regularly use content sourced from the social Web, but is available
for people who want to do it. The organization also regularly circulates
guidance notes to all staff, and there is a verification wiki on the Intranet
with examples of best practice. The BBC is currently re-evaluating training
needs in this area, with an acknowledgement that there are specific needs
around social newsgathering, verification, and rights that might need to be
included in a new training course. ARD offers regular workshops for staff
with external experts to ensure their journalists are kept up to date with
verification techniques.
But these examples were the exception. The norm is on-the-job training. It
was, however, noticeable that in most organizations where we interviewed
staff, there was a shared desire for additional training around verification.
One of the main issues we encountered was people in senior management
positions for whom social media was not a prominent tool when they last
worked on news desks. They didn't know the specific skills necessary for
journalists working with UGC, whether how to effectively search social
platforms for content, verify that content, effectively seek permissions

publicly from people caught up in breaking news situations, or ensure
rights have been cleared effectively for distribution to partners or via
syndication deals.
There are obvious barriers to this type of training, cost being a major one.
Many newsrooms talked about a lack of money for offering training to staff.
Another issue is the pace at which the social media landscape is shifting.
As one journalist mentioned, ``I'm not sure how useful set training pieces
would be for us because everything is changing so quickly. I mean a lot of
the stuff that we're training people on is really focused on practical steps:
‘Do this and do this and do this.' Those steps change so often that we kind
of need to come up with a way that if there are resources, they need to be
almost live.''
\section{Conclusions}
The phenomenon of such large amounts of user-generated content being
shared on social networks means that newsrooms face a number of new
types of responsibilities. These can include consideration for the health and
safety of people who are not trained journalists but are sometimes putting
their lives in danger to capture pictures, or consideration for the welfare of
their own journalists who are watching this content for long periods of time.
As one former journalist wrote, everyone now has access to these images,
and we don't know what the consequences might be:
I know from sitting on a night shift, taking in pictures from plane
crashes and bombs and things, I cannot forget what I saw; horrible
things and the sounds of cameramen being sick and awful, awful
sights that are seared in my memory. And I remember my son saying
he'd looked at something on Twitter and said, ``Mum, I can't bleach
my brain. I can't get that out of my head now. It's just too horrible.
And he's not young. But for young people who could come across
this stuff, the emotional blunting of trauma is not to be underestimated.
And then it just becomes normal, seeing horrific beheadings
being shared on Facebook and thinking that's normal. So I worry that
the psychological trauma is underestimated, and who knows what
kind of an insidious effect it's going to have long-term on people.''
There is certainly a great deal of discussion taking place within newsrooms
around these issues, and there are initiatives such as the Social Newsgathering
group within the Online News Association considering these topics.
Academic research is demonstrating empirically what people have suspected
for a while in terms of vicarious trauma. Awareness is being raised,
which can only be a good thing.
Deborah Rayner of CNN raised one particularly interesting point, however,
along a similar vein: the impact of trauma on the overall news agenda. She
argued, ``I think actually the bigger worry is that when you see more horror
and gore… people are desensitized. You start to lack empathy, and as a
journalist or editor you don't support a story just because [you] don't understand
the real human impact of it, because [you've] just seen it so often.''

When you see news desks and audiences struggling with the complexity
of the Syrian conflict, it makes you wonder whether the fatigue people are
experiencing is actually caused by a form of desensitization.

\chapter{Overall Conclusions}
This research project was designed to answer two key research questions:
1) How is UGC used by broadcast news organizations, on air as
well as online?
2) Does the integration of UGC into output cause any particular
issues for news organizations? What are those issues and how do
they handle them?
The data revealed that, of the 38 newsrooms included as part of this research,
only two had never used UGC in their output. For 24-hour news channels,
UGC is integrated into output almost on a daily basis. On average, the channels
used 11 pieces of UGC per day. Al Jazeera Arabic was an outlier, using
50 pieces per day.
For national bulletins, the reliance on UGC is lower. When it is used, it has
mostly been sourced by a news agency. Overall, UGC is used when other
pictures aren't available; whether that's from a conflict zone where journalists
can't enter, or eyewitness footage from a breaking news event.
While it is useful to have this benchmark in terms of how much UGC is
being used within the industry, the amount of UGC being broadcast on air
and integrated online was not necessarily surprising. What did surprise us
was the amount of UGC that was not labeled or credited. UGC was treated
like any other footage.

Crediting involves naming the uploader. We use the term labeling to
describe the different ways news organizations acknowledge that someone
unconnected to the newsroom created the content. Only 26 percent of
UGC broadcast on air was labeled. Online this figure reached 70 percent.
Only 16 percent of the UGC broadcast during our three-week sample
was given an onscreen credit by the news organizations. In comparison,
37 percent of content online was given a credit by the broadcaster (simply
embedding content was not included as credit added by the broadcaster).
We acknowledge that the process of embedding does mean that the information
about the uploader is made visible.
Our data certainly showed more similarities than differences across television
and Web output, with troubling practices across both platforms. The
best use of UGC was online, mostly because the Web provides opportunities
for integrating UGC into news output like live blogs and topic pages.
The interviews we carried out bought the quantitative data alive, lending
a vital explanatory layer to the findings. Journalists who work with UGC
every day were not surprised by our results, and were able to describe in
detail the processes through which UGC has to travel—providing necessary
context to topline figures. It was noticeable that the very low levels of
crediting and labeling did shock everyone with whom we spoke.
On many of the other topics, however, there wasn't consensus. We were regularly
surprised by the way different journalists talked about the issues raised
by the integration of UGC and how their newsrooms handled these issues.
Overwhelmingly, managers talked about UGC in very different ways than
those who work with it every day. Managers tended to be disconnected
from the reality of the everyday work that is involved with discovering, verifying,
and clearing the rights for UGC. Therefore, in many newsrooms the
staff who work daily with UGC does not receive specific support; whether
that's training in advanced verification techniques, the development of
mandatory crediting guidelines, updates to editorial policies about labeling

UGC based on the agenda of the uploader, or the necessary improvements
to Media Asset Management systems to protect the metadata connected
to UGC.
When we started our research, we interviewed 10 senior managers from
national newsrooms at a large broadcast news conference, and the answers
we heard suggested that UGC is always fully verified, that permission is
always sought, and crediting is systematically given on screen.^{\href{#endnotes}{39}} If this
research had not included a quantitative element, or included interviews
with journalists who work with UGC every day, the conclusions we could
have drawn would have been entirely different. The lack of knowledge
around the specifics of UGC is not surprising. Senior managers in newsrooms
today have never had to face the reality of trying to lead a bulletin
with a picture uploaded to Twitter. The specific skills required to work with
UGC have been developed by certain journalists over the past five years
through necessity. The complexity of these skills are not yet necessarily recognized
at a managerial level.^{\href{#endnotes}{40}}
Our overall conclusion from this research is that news managers need to
understand the full implications of integrating UGC, and do so quickly. We
say this particularly with regard to the impact on their staff, their audiences,
and the people who are creating the content in the first place.
It was not surprising that so many interviews used the term ``Wild West''
to describe the current landscape. A lack of precedent, deliberately vague
terms and conditions used by social networks, and ignorance on the part
of uploaders cause real confusion. But the speed at which this landscape
is shifting means that all journalists, editors, and managers have to understand
this world, and keep up with the pace of change. The most important
issues include the following:

1. Unless crediting practices improve, an uploader will take a news
organization to court for using content, either without permission,
or because he or she was not attributed. The results of such a case
would have wide-reaching implications for the news industry.
2. The current way that UGC is being discovered and used will change
very soon. More and more high-value UGC is being licensed very
quickly after breaking news events. If newsrooms want to use this
content (which are often the only pictures available from a breaking
news event), they will have to pay.
3. The processes needed to verify digital content require specific
knowledge and skills. While traditional journalism techniques are
crucial to the process of verification, they have to be combined with
an understanding of specific tools and practices. It's not necessary
for all staff members to master forensic verification techniques, but
basic verification knowledge should be required. Certain staff in all
newsrooms should be able to independently identify an uploader and
run a full analysis on his or her digital history, as well as confirm
the date and the location of a piece of video, a photo, a Twitter or
Facebook status, or a blog post. Certainly the output editors who
make the final decision about using a piece of UGC need to understand
that verification is a process—not a true/false distinction—and
they need to know which questions to ask of the producers who have
undertaken the process.
4. News organizations' reliance on agencies to discover and verify
UGC is very surprising. News managers therefore need to understand
the differences between the practices at different agencies
around UGC in order to ask appropriate questions about the provenance
of a piece of UGC, and the verification checks that have been
undertaken. Journalists also need to know how to recognize that a

piece distributed by agencies is UGC through dopesheets so they
know they are indeed working with UGC and can add additional
labels to ensure maximum transparency for the audience.
5. An academic study by Professor Anthony Feinstein and colleagues
(to be published in the Journal of the Royal Society of Medicine very
soon)^{\href{#endnotes}{41}} compares self-reported psychological measures of journalists
who work on domestic desks with journalists who work on UGC
desks and frequently in conflict zones. The main feature to emerge
from this study was that frequency of exposure to UGC, independently
and consistently, predicted multiple indices of psychopathology,
be they related to anxiety, depression, PTSD, or alcohol
consumption. This research will require news organizations to take
their responsibility toward staff working with UGC on a daily basis
seriously, and to institute working practices to minimize risk.
6. News organizations' clarity in their public calls to action is a crucial
element of the training needed for staff working with UGC. As
more and more UGC is captured at news events, without clear advice
to people on the ground about how to stay safe, and clear language
that emphasizes that newsrooms will not use content that has been
captured by people putting themselves in danger or committing an
illegal act, it is very likely that tragedy involving someone attempting
to capture footage ``for'' a news organization is on the horizon.
7. For some journalists, their roles will change. They are no longer
the sole bearers of truth, as more space opens to allow people to
tell their own stories directly. News organizations will have to decide
how to manage this change. As one interviewee argued:

I think [journalists] find it very, very hard to let go. They don't
understand what their job is if they allow people to tell their story.
But of course their job is fundamental, their analysis is fundamental,
it's fundamental to have journalists take all these people's stories
and make some sense of them. But I think journalists should
be ready to shift the power a bit and make this type of journalism
mainstream. UGC is still thought of as separate somehow. It's kind
of, ``Well, we'll add a little bit, because it shows that we care,'' rather
than it being absolutely fundamental to the news organization.
There is a critical need for more audience research on this topic. Almost
nothing is known about how audiences consider the verification of UGC,
how they feel about onscreen credits or, indeed, how they value the use of
UGC in reports. In addition, very little research has been undertaken with
those people who are caught up in breaking news events and post content to
social networks, only to be bombarded by requests from journalists regarding
usage. A number of our interviewees asked whether audience research
about UGC existed, and went on to admit that it didn't matter what recommendations
this report suggested, nothing would change unless evidence
showed that audiences cared.
We would also like to expand this research by studying whether similar patterns
appear within newspaper organizations. Large newspaper sites are
investing heavily in online video, and the role UGC plays in this context is
important to understand in greater detail.^{\href{#endnotes}{42}}
We end this report with some suggestions and recommendations. These are
designed to be conversation starters—hopefully a catalyst for some industry-
wide meetings where journalists, representatives from social networks,
lawyers, and educators can come together to discuss the reality of the situation,
and work toward taking steps to improve current practices.

Throughout this process we have frequently reminded ourselves that it's
very easy for outsiders to write a report full of suggestions that are ultimately
unworkable within the everyday context of a pressured breaking
news environment. As part of the interview process we talked to journalists
about practical solutions and asked them what they would find useful.
Most said they simply wanted to learn more about how other news organizations
handled UGC. This request and other suggestions are reflected in
the recommendations. It is important to note that the Online News Association
is supporting a group of journalists who have convened around the
issue of social newsgathering. Led by Fergus Bell and Eric Carvin of the AP,
the group is facilitating industry-wide discussion about many of the issues
raised within this report.^{\href{#endnotes}{43}}
While more audience research is certainly needed, and will feed into these
discussions, it is important that these conversations start sooner rather
than later. Look what has happened in five years. Imagine how a report 10
years beyond the Hudson plane-landing might read.

\chapter{Recommendations}
In light of the findings of this research, we have made the
following recommendations.
\section{Recommendations for the Entire News Industry:}
\begin{itemize}
\item Newsrooms should invest properly in UGC and stop treating
it solely as a source of breaking news footage. Building a
community of trust with the audience creates valuable opportunities
for organizations to differentiate their output from that
of their competitors.
\item Crediting guidelines should be implemented throughout every
organization and at every step of the process, from the first view
of a piece of UGC right up to the point at which it goes to air
or online. All staff needs to be aware that failure to credit could
result in lawsuits or extra payment to uploaders.
\item All staff should be taught basic digital verification skills. Within
each newsroom, there should be a core group of journalists who
can undertake forensic-level verification analysis. It is just as
important that these skills are mastered by producers as it is by
the editors who make final decisions about output.
\item All newsrooms must develop an awareness of the potential risks
to citizen journalists filming and uploading content, and the need
to advise uploaders not to jeopardize their personal safety. Newsrooms
should also implement clear guidelines to all staff about
when it is safe to contact uploaders or use their details on air.
\item Industry-wide guidelines should be created to ensure total transparency
with the audience about who filmed UGC footage that is
put onscreen (whether activists, eyewitnesses, NGOs, et al.).
\item Newsroom technology should be developed to enable credits to
be burnt onto UGC video during transcoding as the video enters
the newsroom Media Asset Management system. This includes
video ingested from agencies and other partners.
\item Mechanisms and best-practice guidelines should be instituted
for managers and staff regularly working with UGC in order to
prevent vicarious trauma. On an individual level, this includes
specific advice about the effects of upsetting UGC video and
vicarious trauma, and access to anonymous counseling. On
a managerial level, this includes advice such as rotating
shift patterns and an understanding of how to spot signs of
vicarious trauma.
\item Ethical codes should be developed around what content should
or should not be used from the social Web, particularly when the
uploader would not have expected it to be used by a news organization,
or if it is likely the uploader had little understanding of
the privacy settings of a particular network.
\end{itemize}

\section{Recommendations for News Agencies:}
\begin{itemize}
\item When distributing UGC to clients, news agencies should ensure
the associated information sheet contains clear details about
the uploader and the steps that have been taken to verify the
content. Agencies should also inform clients about the quantity
of UGC used through their watermarking statistics so as to
increase awareness about UGC use among smaller broadcasters.

\section{Recommendations for Social Networks:}
\item Social networks should create a common, transparent standard
for metadata that accompanies all content on their platforms.
This should include clear and consistent time and date stamps,
geo-location, and uploader information.
\item Social networks should work toward a shared 24-hour news
license. Similar to Creative Commons, this would empower
users to permit news usage of their content either when signing
up for a network or when uploading individual pieces of content.
\item Social networks should develop an automated messaging system
that alerts individual users when their content is embedded onto
a news site.
\item Social networks, with the aid of educational institutions, should
strive to educate uploaders about their rights under copyright
law, particularly geographical nuances. Newsrooms working in
territories where ``fair dealing for the purposes of reporting'' is
used need to ensure this is not being claimed as a way of using
user-generated content inappropriately.
\end{itemize}

\section{Recommendations for Newsrooms and Journalism Schools}
\begin{itemize}
\item Newsrooms and journalism schools should create specific and
evolving training around working with UGC to ensure (1) senior
staff fully understand the challenges posed by this type of content
and (2) the practical and ethical issues related to UGC are
integrated into the mandatory elements of journalism curricula.
This training should encompass ethical public newsgathering,
advanced verification techniques, copyright law, crediting
and labeling content, wording calls to action, and vicarious
trauma prevention.
\end{itemize}

\chapter{Endnotes}

1 Anthony de Rosa, ``Disconnect Between Traditional Media and UGC,'' 12 Aug. 2013,
www.antderosa.com/2013/11/30/the-disconnect-between-traditional-media-and-ugc/. \\
2 See http://quernstone.com/archives/2010/11/user-generated.html and http://rooreynolds.
com/2009/05/07/alternatives-to-ucg/ for two discussions about the widely shared opinion that
the phrase UGC is unpopular and actually unhelpful. \\
3 C. Wardle and A. Williams, ``ugc@thebbc: Understanding its Impact Upon Contributors, Noncontributors
and BBC News,'' AHRC Knowledge Exchange, 2008, www.bbc.co.uk/blogs/legacy/
knowledgeexchange/cardiffone.pdf.\\
4 The five days we analyzed were: November 27, 2013; November 30, 2013; December 2, 2013;
December 5, 2013; and December 11, 2013. The website capture software failed on three of the 40
occasions that are included in this sample. For those days, the previous day was analyzed.\\
5 This report analyzes the role of social media in the Syrian conflict and demonstrated the scale
of activist networks uploading content. See M. Lynch, D. Freelon, and S. Aday, ``Syria's Socially
Mediated Civil War,'' United States Institute of Peace, Jan. 2014, www.usip.org/ publications/syrias-
socially-mediated-civil-war.\\
6 Average refers to the mean average.\\
7 N. Thurman and N. Newman, ``The Future of Breaking News Online? A Study of Live Blogs
Through Surveys of Their Consumption, and of Readers' Attitudes and Participation,'' Journalism
Studies, 2014, http://dx.doi.org/10.1080/1461670X.2014.882080.\\
8 The UGC video was actually submitted directly to GuardianWitness, the UGC initiative from
the Guardian. The video was so dramatic that it was syndicated to a number of different news
organizations, including the BBC. It is also worth noting that the video was not described as UGC
or credited.\\
9 Two non-news features on CNN International included very high numbers of UGC.\\
These were not included in this table as we wanted to compare stories that appeared on more
than one channel.\\
10 A nice roundup of recent discussion about this subject appears as part of Neiman Journalism
Lab's weekly update here: www.niemanlab.org/2013/12/this-week-in-review-questions-onjournalists-
handling-of-nsa-files-and-the-value-of-viral-content/.\\
11 The only content that could be described in this category was a two-minute roundup showcasing
online responses to Miley Cyrus' appearance at the American Music Awards in front of a giant
lip-syncing cat. It showed the audience's responses, which appeared in tweets, Vine videos, and
pictures posted to social networks.\\
Amateur Footage: A Global Study of User-Generated Content in TV and Online News Output
132\\
12 The Economist, ``Drones Often Make News. They Have Started Gathering it, Too,'' 29 Mar. 2014,
www.economist.com/news/international/21599800-drones-often-make-news-they-havestarted-
gathering-it-too-eyes-skies?fsrc=scn/tw/te/pe/eyesintheskies.\\
13 The role of social media, in terms of documenting human rights abuses, is discussed in this article:
C. Koettl, ``The YouTube War: Citizen Videos Revolutionize Human Rights Monitoring in Syria,''
MediaShift, 18 Feb. 2014.\\
14 J. Harkin, et al., ``Deciphering User Generated Content in Transitional Societies: A Syria Coverage
Case Study,'' Internews Center for Innovation and Learning, 2012, www.internews.org/sites/
default/files/resources/ InternewsWPSyria_2012-06-web.pdf.\\
15 This is discussed is detail by Chris Paterson in his chapter, ``Global Television News Services,'' in
Media in Global Context, ed. A. Sreberney-Mohammadi et al. (London: Arnold, 2003), 145–160.\\
16 Storyful was acquired by News Corp in December of 2013. See ``NewsCorp Acquires Social News
Agency Storyful,'' NewsCorp Press Release, 20 Dec. 2013, http://newscorp.com/2013/12/20/
news-corp-acquires-social-news-agency-storyful/.\\
17 EBU members, http://www3.ebu.ch/members, accessed 7 May 2014.\\
18 There have been a number of articles about the rise of the hoax, and simultaneous interest by the
audience when these are debunked. See www.buzzfeed.com/charliewarzel/2014-is-the-year-ofthe-
viral-debunk and www.buzzfeed.com/tomphillips/14-incredible-but-fake-viral-imagesand-
the-twitter-account.\\
19 In January of 2014, The Verification Handbook was published by the European Journalism Centre. It
was edited by Craig Silverman and includes detailed explanations of how to verify digital content. It
is available at http://verificationhandbook.com.\\
20 BBC Trust (2012) and BBC Trust (2013).\\
21 This is the title of a study by Nicola Bruno, published in 2011.\\
22 M. Ingram, ``The Future of News Isn't About Breaking News Scoops, It's About Credibility and
Trust,'' GigaOm, 7 May 2014, https://gigaom.com/2014/05/07/the-future-of-media-isnt-aboutbreaking-
news-scoops-its-about-credibility-and-trust/.\\
23 ``YouTube & News: A New Kind of Visual Journalism,'' Pew Research Center: Project for Excellence
in Journalism, 2012, www.journalism.org/analysis_report/youtube_news.\\
24 There is also a significant problem caused by the lack of legal precedent. One of the most confusing
elements is that within YouTube's Terms and Conditions, downloading video is prohibited.
However, it is in the interest of the platform that high-quality video to be seen on news broadcasts
as it might drive traffic back to the site, and therefore increase view counts and revenue. This leads
to confusion around what is legal and illegal.\\
25 The ethics of using content published on social networks was raised after BuzzFeed published a
series of tweets sent by women about the clothes they had been wearing when they were sexually
abused. It led to a number of discussions about the ethics of embedding content that was not
originally created with the intent of ending up on a news site. This article by Slate provides a
useful roundup of the debate. www.slate.com/articles/technology/technology/2014/03/twitter_journalism_private_lives_public_speech_how_reporters_can_ethically.html.\\
26 http://scienceblog.cancerresearchuk.org/2014/03/25/nomakeupselfie-some-questionsanswered/,
accessed 12 May 2014.\\
27 www.rts.org.uk/winners-tja, accessed 12 May 2014.\\
28 S. Myers, ``How Citizen Journalism has Changed Since George Holliday's Rodney King Video,''
Poynter, 3 Mar. 2011, www.poynter.org/latest-news/top-stories/121687/how-citizenjournalism-
has-changed-since-george-hollidays-rodney-king-video/.\\
29 M. Little, ``A Public License for Online News Video,'' Storyful blog, 16 May 2013, http://blog.
storyful.com/2013/05/16/a-public-license-for-online-news-video.\\
30 Final checks against the data included in the first phase of this report highlighted a couple of
discrepancies. We apologize for that and can ensure these numbers are correct.\\
31 J. Guillame, ``33 Hilarious Reactions to NewsCorp's Insane Watermarking of Packer Punch-up,''
BuzzFeed, 5 May 2014, www.buzzfeed.com/jennaguillaume/reactions-to-news-corpswatermark-
on-packer-punchup-pics.\\
32 A dopesheet is metadata that agencies send to accompany a video package that describes the
content of the video in written form. It includes a written shotlist, storyline, dates, restrictions on
use, and crediting requirements.\\
33 See www.hoffmanlawfirm.org/Publications/1-16-13-PRESS-RELEASE.docx, accessed 12 May 2014.\\
34 Jenny Hauser writes about the importance of transparency regarding who is uploading content
and their motivations for doing so here: http://blog.storyful.com/2014/03/30/a-place-forprofessional-
journalism-in-the-age-of-the-citizen-journalist.\\
35 This event was discussed by The New York Times in its Lens Blog; Reuters issued a reply a few days
later. See http://lens.blogs.nytimes.com/2014/03/13/questions-about-news-photographersin-
syria-arise-after-freelancers-death/?_php=true&_type=blogs&_r=0 and http://petapixel.
com/2014/03/16/reuters-responds-accusations-leveled-agency-new-york-times/.\\
36 Ofcom is the independent regulator and competition authority for the UK communications
industries, www.ofcom.org.uk/.\\
37 J. Halliday, ``Woolwich Attack Video: TV Broadcast Prompts 800 Complaints,'' Guardian, 23 May
2013, www.theguardian.com/media/2013/may/23/woolwich-attack-video-tv-complaints.\\
38 A. Feinstein, B. Audet, and E. Waknine, ``Witnessing Images of Extreme Violence: A Psychological
Study of Journalists in the Newsroom,'' Journal of the Royal Society of Medicine, forthcoming 2014.\\
39 It is worth noting that all but one of our interviewees was from a news organization not included in
our quantitative analysis, and the answers given by Take Kusaba of NHK mirrored the use of UGC
we encountered during our research.\\
40 Research carried out by City University in the United Kingdom involved interviewing journalists
about the type of technological support they would like to help them gather news using social
networks. Discovery and verification technology was rated highly. See S. Schifferes, et al.,
``Identifying and Verifying News Through Social Media: Developing a User-centred Tool for
Professional Journalists,'' Digital Journalism, 2014.\\
41 A. Feinstein, et al., 2014.\\
42 This article explains the main aspects of the transition to digital, focusing on online video. P.
Grabowicz, ``The Transition to Digital Journalism,'' Berkeley School of Journalism, 30 Mar. 2014,
http://multimedia.journalism.berkeley.edu/tutorials/digital-transform/video.\\
43 An introduction to the main concerns of the group are detailed in this article: ``Social
Newsgathering: Charting an Ethical Course,'' Online News Association, 6 Mar. 2014, http://
journalists.org/2014/03/06/social-newsgathering-charting-an-ethical-course/.\\

\chapter{Further Reading}

K. Anderson, ``Clear Editorial Goals Essential to Effective UGC Strategy,''
22 Mar. 2013, www.kbridge.org/clear-editorial-goals-essential-toeffective-
ugc-strategies/.\\
BBC Trust, ``A BBC Trust Report on the Impartiality and Accuracy of
the BBC's Coverage of the Events Known as the ‘Arab Spring,''' 2012,
http://downloads.bbc.co.uk/bbctrust/assets/files/pdf/our_work/
arabspring_impartiality/arab_spring.pdf.\\
BBC Trust, ``A BBC Trust Report on the Impartiality and Accuracy of
the BBC's Coverage of the Events Known as the ‘Arab Spring': Follow Up
Report,'' 2013, http://downloads.bbc.co.uk/bbctrust/assets/files/pdf/
our_work/arabspring_impartiality/follow_up.pdf.\\
C. Beckett and R. Mansell, ``Crossing Boundaries: New Media and
Networked Journalism,'' Communication, Culture & Critique, 1:1
(2008): 92–104.\\
N. Bruno, ``Tweet First, Verify Later? How Real-time Information is
Changing the Coverage of Worldwide Crisis Events,'' Reuters Institute
Fellowship Paper: University of Oxford, 2011, https://reutersinstitute.
politics.ox.ac.uk/fileadmin/documents/Publications/fellows_papers/2010-2011/TWEET_FIRST_VERIFY_LATER.pdf.\\

D.L. Cade, ``Reuters Responds to Accusations Leveled at the Agency in
the New York Times Lens Blog,'' PetaPixel, 16 Mar. 2014, http://petapixel.
com/2014/03/16/reuters-responds-accusations-leveled-agency-newyork-
times/.\\
E. Carvin and F. Bell, ``Social Newsgathering: Charting an Ethical
Course,'' Online News Association, 6 Mar. 2014, http://journalists.
org/2014/03/06/social-newsgathering-charting-an-ethical-course/.\\
M. Coddington, ``This Week in Review: Questions on Journalists'
Handling of NSA files, and the Value of Viral Content,'' Nieman Journalism
Lab, 6 Dec. 2013, www.niemanlab.org/2013/12/this-week-in-reviewquestions-
on-journalists-handling-of-nsa-files-and-the-value-ofviral-
content/.\\
Anthony de Rosa, ``Disconnect Between Traditional Media and UGC,'' 12
Aug. 2013, www.antderosa.com/2013/11/30/the-disconnect-betweentraditional-
media-and-ugc/.\\
The Economist, ``Drones Often Make News. They Have Started
Gathering it, Too,'' 29 Mar. 2014, www.economist.com/news/
international/21599800-drones-often-make-news-they-have-startedgathering-
it-too-eyes-skies?fsrc=scn/tw/te/pe/eyesintheskies.\\
J. Estrin and K. Shoumali, ``Questions About News Photographers in Syria
Arise After Freelancer's Death,'' New York Times Lens Blog, 13 Mar. 2014,
http://lens.blogs.nytimes.com/2014/03/13/questions-about-newsphotographers-
in-syria-arise-after-freelancers-death/.\\
A. Feinstein, B. Audet, and E. Waknine, ``Witnessing Images of Extreme
Violence: A Psychological Study of Journalists in the Newsroom,'' Journal
of the Royal Society of Medicine, forthcoming 2014.\\

P. Grabowicz, ``The Transition to Digital Journalism,'' Berkeley School of
Journalism, 30 Mar. 2014, http://multimedia.journalism.berkeley.edu/
tutorials/digital-transform/video.\\
J. Guillame, ``33 Hilarious Reactions to NewsCorp's Insane Watermarking
of Packer Punch-up,'' BuzzFeed, 5 May 2014, www.buzzfeed.com/
jennaguillaume/reactions-to-news-corps-watermark-on-packerpunchup-
pics.\\
M.T. Hänska-Ahy and R. Shapour, ``Who's Reporting the Protests?''
Journalism Studies, 13:1 (2012): 1–17.\\
J. Harkin, et al., ``Deciphering User Generated Content in Transitional
Societies: A Syria Coverage Case Study,'' Internews Center for Innovation
and Learning, 2012, www.internews.org/sites/default/files/resources/
InternewsWPSyria_2012-06-web.pdf.\\
J. Hauser, ``A Place for Professional Journalism in the Age of the
Citizen Journalist,'' Storyful blog, 30 Mar. 2014, http://blog.storyful.
com/2014/03/30/a-place-for-professional-journalism-in-the-age-ofthe-
citizen-journalist.\\
A. Hermida and N. Thurman, ``A Clash of Cultures: The Integration of
User-generated Content Within Professional Journalistic Frameworks at
British Newspaper Websites,'' Journalism Practice 2(3) (2008): 343–356.
A. Hess, ``Is All of Twitter Fair Game for Journalists?,'' Slate, 19 Mar. 2014,
www.slate.com/articles/technology/technology/2014/03/twitter_journalism_private_lives_public_speech_how_reporters_can_ethically.html.\\
A.M. Jönnson and H. Örnebring, ``User-generated Content and the News:
Empowerment of Citizens or an Interactive Illusion?'' Journalism Practice,
5:2 (2011): 127–144.\\

C. Koettl, ``The YouTube War: Citizen Videos Revolutionize Human Rights
Monitoring in Syria,'' MediaShift, 18 Feb. 2014.\\
M. Little, ``A Public License for Online News Video,'' 16 May 2013,
http://blog.storyful.com/2013/05/16/a-public-license-for-onlinenews-
video.\\
M. Lynch, D. Freelon, and S. Aday, ``Syria's Socially Mediated Civil War,''
United States Institute of Peace, Jan. 2014, www.usip.org/ publications/
syria-s-socially-mediated-civil-war.\\
S. Myers, ``How Citizen Journalism has Changed Since George Holliday's
Rodney King Video,'' Poynter, 3 Mar. 2011, www.poynter.org/latest-news/
top-stories/121687/how-citizen-journalism-has-changed-since-georgehollidays-
rodney-king-video/.\\
C. Paterson, ``Global Television News Services,'' in Media in Global
Context, eds. A. Sreberney-Mohammadi et al. (London: Arnold, 2003),
145–160.\\
S. Paulussen and P. Ugille, ``User Generated Content in the Newsroom:
Professional and Organizational Constraints on Participatory Journalism,''
Westminster Papers in Communication and Culture, Vol. 5:2 (London:
University of Westminster, 2008), 24–41. https://www.westminster.ac.uk/_data/assets/pdf_file/0005/20021/003WPCC-Vol5-No2-Paulussen_Ugille.pdf.\\
``YouTube & News: A New Kind of Visual Journalism,'' Pew Research
Center: Project for Excellence in Journalism, 2012, www.journalism.org/
analysis_report/youtube_news.\\
T. Phillips, ``14 Incredible But Fake Viral Images and the Twitter Account
Debunking the Picspammers,'' BuzzFeed, 10 Feb. 2014, www.buzzfeed.
com/tomphillips/14-incredible-but-fake-viral-images-and-thetwitter-
account.\\
R. Reynolds, ``Alternatives to UGC,'' 7 May 2009,
www.rooreynolds.com/2009/05/07/alternatives-to-ucg/.\\
J. Sanderson, ``User Generated Content,'' 26 Nov. 2010,
www.quernstone.com/archives/2010/11/user-generated.html.\\
S. Schifferes, et al., ``Identifying and Verifying News Through Social Media:
Developing a User-centred Tool for Professional Journalists,'' Digital
Journalism, 2014, http://dx.doi.org/10.1080/21670811.2014.892747>.\\
C. Silverman, ed., ``The Verification Handbook,'' 2014,
http://verificationhandbook.com.\\
N. Thurman and N. Newman, ``The Future of Breaking News Online?
A Study of Live Blogs Through Surveys of Their Consumption, and of
Readers' Attitudes and Participation,'' Journalism Studies, 2014,
http://dx.doi.org/10.1080/1461670X.2014.882080.\\
C. Wardle and A. Williams, ``ugc@thebbc: Understanding its Impact Upon
Contributors, Non-contributors and BBC News,'' AHRC Knowledge
Exchange, 2008, www.bbc.co.uk/blogs/legacy/knowledgeexchange/
cardiffone.pdf.\\
C. Wardle and A. Williams, ``Beyond User-generated Content: A
Production Study Examining the Ways in Which UGC is Used at the BBC,''
Media, Culture & Society, 2010, 781–799.\\
C. Warzel, ``2014 is the Year of the Viral Debunk,'' BuzzFeed, 23 Jan. 2014,
www.buzzfeed.com/charliewarzel/2014-is-the-year-of-theviral-
debunk.\\

\chapter{About the Authors}
Claire Wardle has a Ph.D. in Communication from the Annenberg School
for Communication at the University of Pennsylvania. She started her career
at Cardiff University, where she undertook a year-long research project on
UGC at the BBC. In 2009, she took what she thought would be a short break
to design the social media training programme for BBC News in 2009. Since
then she has been training journalists around the world on social newsgathering
and verification, including a year working with the social media news
agency Storyful.

www.clairewardle.com | @cwardle


Sam Dubberley has over ten years experience in broadcast news. He is an
independent media researcher and adviser—working on a variety of media
projects. He was head of the Eurovision News Exchange from 2010 to 2013,
managing the world's largest exchange of television news content. He was a
bulletin editor for Bloomberg Television.
www.samdubberley.net | @samdubberley


Pete Brown completed his Ph.D. at Cardiff University, School of Journalism,
Media and Cultural Studies in 2013. Since then he has worked independently
on a number of different research projects, including a recent
examination of gender and representation on BBC Local radio.

@beteprown